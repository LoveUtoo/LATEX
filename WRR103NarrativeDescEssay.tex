\documentclass[12pt]{article}
\usepackage[utf8]{inputenc}
\usepackage{amssymb, amsmath, amsthm, changepage, graphicx, caption, subcaption, setspace}

\graphicspath{{./images/}}

\title{The Pencil in the Back}
\author{Nicolas Coballe \\ Daniel Adleman \\ WRR103}

\setlength{\parindent}{5ex}

\begin{document}



\maketitle


\doublespacing

In my grey, japanese, cat pencil case, there exists three distinct mechanical pencils that I have owned for over four years. One of them is broken and unusable--- Why I keep that pencil in my pencil case is still a mystery to me. One of them is my daily driver, a German-manufactured 0.5mm Rotring 500 drafting pencil--- It's a hefty one, having a full brass exterior. The last pencil is a perfectly functional 0.7mm Sakura 127. How many times have I used this 0.7mm Sakura pencil in the last four years? Probably, zero times.

What is wrong with this pencil? Nothing is wrong with this pencil. It's a perfectly fine pencil for sketching and drawing and taking notes and other pencil-based activities. It has a smooth, cylindrical, sleek-matte black body with shiny aluminum tip. As one runs down the exterior of the chassis, its cylindrical shape tapers off into a more discrete polygonal prism, featuring ring notches where one grips the pencil. The cap, the button that dispenses the pencil graphite, is also a shiny aluminum colour, matching the metal pencil clip--- I have always wondered if people actually use pencil clips or pen clips. Most of the times I see people use it, its in some sit-com on TV, where typically the nerdy character puts a clips a pen into his breast-pocket, and the joke eventually leads to the nerd not realising that pen-ink is spilling into his breast-pocket.

As one can see, the Sakura 127 is a magnificent pencil. I picked it up at an art store near my house around four or five years ago. I didn't really care for the pencil, only being interested in the pens that came with the pencil; although, I did give the pencil a bit of use after I acquired it because I required a pencil for pencil-based activities prior to owning my Rotring 500. After I purchased my Rotring 500 on \textit{Amazon} about a month or two afterwards, I never used the Sakura again. The only reason I don't use it is because the Rotring 500 is \textit{better}. Better in build quality; better in weight; better in feel. It doesn't diminish the capabilities of the Sakura 127; however, why would I use a good, albeit inferior pencil, when I can simply use a similarly good superior pencil. Thus, logically, it only makes sense to keep the Sakura 127 to serve as a pencil that I lend to others that need it.

Thus, this pencil has been passed around a lot of people's hands. It's been shared with dozens of highschool acquaintances, yet miraculously it maintained its functionality and never got lost or damaged. I suppose I have good taste in acquaintances if my spare pencil has been treated so well over the last four years--- the only "damage" to the pencil is just the silver text saying, "0.7 Sakura 127" that has started to fade off the body. Nonetheless, my Sakura 127 has been in great condition until recently.

After moving to Toronto to attend university, it was a strange transition. I'm originally from Ottawa, Ontario, so not so-far from Toronto in terms of physical distance and culture. However, not everyone on University of Toronto campus is from around Ontario. I realise now that I'm abundantly fortunate to be from Ontario because I am now aware that Ontarians, and maybe even more generally, Canadians, are quite respectful of other people's belongings. That is probably why during highschool I managed to preserve my Sakura Pencil, despite the vast amount of different people who used it. Though a caveat of Canadian kindness is that if someone isn't from Canada, they are not guaranteed to be greater than or equally as respectful towards other's property.

A month prior to me writing this, I was in a private study-room in Robarts Library with some fellow math students. We were discussing that week's problem set and working through the problems. One of my classmates somehow forgot to bring one of the only things required to do math homework: a writing utensil. So again, my Sakura 127 came in handy, and I lent him the pencil before starting work myself. We worked for about 30 minutes, and afterwards we all have to leave. I requested the return of my pencil, but to my despair, the clip of the pencil was irreversibly bent upwards. I tried as much as I could to bend it back to its original position, but it never wavered from looking like a one-spocked umbrella. Obviously, now the pencil is awkward to write with. I firmly asked the perpetrator why he bent my pencil's clip, but he gave a non-answer, saying, "I don't know." Although bending a pencil doesn't seem like a huge problem, I was outraged at the lack of respect that this person treated my pencil with.

I would be okay if somehow the pencil got damaged through writing or proper use, but he damaged the pencil clip, which for all reasonable situations, serves no purpose. Thus, he must have went out of his way to damage my pencil--- again, without clear \textit{mens rea}. And when he returned it, he did not even have the respect to apologize. I was very vocal about this vandalization, not because I particularly cared about this pencil, but because what he did was extremely disrespectful on principle. I have always lived in a household where it was expected that if one was borrowing something, one must return in equal or better condition, and if one failed to do so, one must be sincerely apologetic. My classmate showing a wanton of disregard for my property unsettled me.

In retrospect, I may have been more flustered than what is considered socially acceptable, so that incident is not particularly pleasant to think about; however, I am forced to relive the apathetic treatment of my pencil every time I open my pencil case, seeing the bent Sakura 127. Its not horrific, and its not frightening, yet its just an unpleasantry that I would rather abstain from thinking about. I do find it bizarre that because of this interaction, I considered editing the contents of my pencil case for the first time in four years, and removing my Sakura 127 rather than an actual non-functioning, broken pencil. Regardless, I am indecisive when it comes to inconsequential decisions, so the Sakura 127 remains in my pencil case.

\end{document}