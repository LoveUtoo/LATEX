\documentclass[11pt]{article}
\usepackage[utf8]{inputenc}
\usepackage{amssymb, amsmath, amsthm, changepage}

\title{MAT157 Problem Set 2}
\author{Nicolas Coballe}

\newcommand{\bproof}{\begin{proof}
$ $ \\
\begin{adjustwidth}{3em}{0pt}
}

\newcommand{\eproof}{\end{adjustwidth}
\end{proof}}

\begin{document}

\maketitle
\begin{flushleft}

1. a) $\forall x \in F, \ \exists y \in F : \forall z \in F, \ z \neq x, \ xy = 1 \implies yz \neq 1$ \\
The negation: \\
$\exists x \in F, \ \forall y \in F: \exists z \in F, z \neq x, xy = 1 \text{ and } yz = 1$ \\
Plain English: \\
There exists an element $x$ that for all $y$ such that there exists $z$ with $z \neq x$, $xy = 1$ and $yz = 1$ \\
\bigskip
b) Let $a(x)$ be the angle sum of polygon $x$ and let $H$ be the set of all hyperbolic octagons. \\
$\forall x \in H, \ a(x) > \pi$ \\
The negation:  \\
$\exists x \in H, \ a(x) \leq \pi$ \\
Plain English: \\
There exists a hyperbolic octagon with an angle sum less than or equal to $\pi$. \\
\bigskip
c) Let $Q$ be the set of all flavors of quarks, $c(x)$ be the charge of quark $x$, and $m(x)$ be the mass of quark $x$. \\
$\forall x,y \in Q, c(x) = c(y) \text{ and } m(x) = m(y)$ \\
The negation: \\
$\exists x,y \in Q, c(x) \neq c(y) \text{ or } m(x) \neq m(y)$ \\
Plain English: \\
There exists two flavors of quarks that do not have the same charge or the same mass. \\
\bigskip
d) Let $S$ be the set of students in this class, $H$ be the set of all homework assignments, $L$ be the set of all lectures, $h(x,y)$ be student $x$ does homework $y$, $l(x,y)$ be student $x$ goes to lecture $y$, and $f(x)$ be the percentage that student $x$ gets on the final exam. \\
$\forall x \in S: h(x,y), \forall y \in H \text{ and } l(x,z), \forall z \in L \implies f(x) \geq 50$ \\
The negation: \\
$\exists x \in S: (h(x,y), \forall y \in H \text{ and } l(x,z), \forall z \in L) \text{ and } f(x) < 50$ \\
Plain English: \\
There exists a student who does all the homework and all the assignments, but scores less than 50 percent on the final exam. \\

\newpage

2. a)

\bproof

$\Rightarrow$ \\
If $a = b$ then $|a - b| = |a - a| = 0 < \epsilon$. \\
\bigskip

$\Leftarrow$ \\
We will take the contrapositive. If $a \neq b$ then $\exists x \in \mathbb{R} \setminus \{ 0 \}$ such that $b = a + x$. Thus, $|a - b| = |a - (a + x)| = |x| > \epsilon$
\eproof

b)

\bproof

Because the distance between $a$ and $b$ is a positive real number, $\forall \epsilon > 0$ we can choose $q \in \mathbb{N}$ arbitrarily large such that $q(b-a) > \epsilon$. Thus we can choose $q$ such that $(qb-qa) > 10$. Using the ceiling function, $ \lceil qa \rceil - qa < 1$. If we take $ \lceil qa \rceil + 1$, the distance $ (\lceil qa \rceil + 1) - qa < 2$. Note that $\lceil qa \rceil + 1$ is a rational number, greater that $qa$ and less than $qb$ because it's distance from $qa$ is less than 10. Thus, $qa < \lceil qa \rceil + 1 < qb$. If we take $x = \frac{\lceil qa \rceil + 1}{q}$, then $a < x < b$.

\eproof
                                                                                                                                                            \newpage
                                                                                                                                                            
                                                                                                                                                         
3. a)

\bproof

$\Rightarrow$ \\
We will prove this using the contrapositive: If $\exists a \in F$ such that $a \cdot a + 1 = 0$, then $F^c$ is not a field. For the sake of contradiction, assume that $F^c$ is a field. Then, take product of two non-zero elements, $(a,1)\cdot (-a,1)$. We can rewrite this as $-a^2 -ai +ai +i^2$. But this is simply $-1(a^2 + 1)$, which is 0. This is a contradiction because two non-zero elements cannot have a product of zero. \\
\bigskip

$\Leftarrow$ \\
We will show that every non-zero element has a well-defined multiplicative inverse. Consider the arbitrary element in $F^c$, $(a,b)$. Then, $(a,b)$ has multiplicative inverse, $(c,d)$ where $c = \frac{a}{a^2 + b^2}$ and $d = \frac{-b}{a^2+b^2}$ (it is very easy to find $c$ and $d$ by multiplying $\frac{1}{a+bi}$ by its conjugate). This inverse is only well-defined, if $a^2 + b^2 \neq 0$. Because $(a,b)$ is non-zero, we have 3 cases: \\
Case 1: $a = 0, \ b \neq 0$. Thus we have $a^2 + b^2 = 0 + b^2$, and since $b^2$ is non-zero, $0 + b^2$ is not zero. \\
Case 2: $a \neq 0, \ b = 0$. This is the same as case 1 except reversed. \\
Case 3: $a,b \neq 0$, Thus if $a^2 + b^2 = 0$, then $\frac{a^2}{b^2} + 1 = 0$, which contradicts our assumption that there does not exists an element in $F$ such that $a \cdot a + 1 = 0$.

\eproof

b)

\bproof

$\Rightarrow$\\
We will prove this using the contrapositive: If $\exists a \in F$ such that $a^2 + a + 1 = 0$, then $F'$ is not a field. For the sake of contradiction, assume that $F'$ is a field. Then, take product of two non-zero elements, $(a+1,1)\cdot (-a,1)$. We can rewrite this as $-a^2 +ai -a +i -ai + i^2$. But this is simply $(-a^2 -a) + (i^2+i) = 1 - 1$, which is 0. This is a contradiction because two non-zero elements cannot have a product of zero. \\
\bigskip

$\Leftarrow$ \\
We will show that every non-zero element has a well-defined multiplicative inverse. Consider the arbitrary element in $F'$, $(a,b)$. Then, we will do a series of steps to find out what the inverse of $(a,b)$ is:
\begin{align*}
(a+bi)\frac{1}{a+bi} = & 1 \\
(a+bi)\frac{(b+ai)}{(a+bi)(b+ai)} = & 1 \\
(a+bi)\frac{(b+ai)}{ab + a^2i +b^2i+abi^2} = & 1 \\
(a+bi)\frac{(b+ai)}{ab + a^2i + b^2i -ab -abi} = & 1 \\
(a+bi)\frac{(b+ai)}{a^2i + b^2i-abi} = & 1 \\
(a+bi)\frac{(b+ai)}{i(a^2 + b^2-ab)} = & 1 \\
(a+bi)\frac{(b+ai)(1+i)}{i(a^2 + b^2-ab)(1 + i)} = & 1 \\
(a+bi)\frac{(b+bi+ai+ai^2)}{(a^2 + b^2-ab)(i + i^2)} = & 1 \\
(a+bi)\frac{(b + bi + ai -a -ai)}{(a^2 + b^2-ab)(-1)} = & 1 \\
(a+bi)\frac{(b-a)+(b)i}{(-a^2 + -b^2+ab)} = & 1 \\
\end{align*}
Thus there is an inverse to $(a,b)$, namely, $(c,d)$ where $c = \frac{b-a}{-a^2 + -b^2+ab}$ and $d = \frac{b}{-a^2 + -b^2+ab}$. This inverse is only well defined, if $ -a^2 + -b^2+ab \neq 0$. Because $(a,b)$ is non-zero, we have 3 cases: \\
Case 1: $a = 0, \ b \neq 0$. Thus we have $-a^2 + -b^2 +ab = 0 + -b^2 = 0$, and since $b^2$ is non-zero, $0  -b^2 +0$ is not zero. \\
Case 2: $a \neq 0, \ b = 0$. This is the same as case 1 except reversed. \\
Case 3: $a,b \neq 0$, Thus if $-a^2  -b^2 +ab = 0$, then $\frac{a^2}{b^2} + \frac{a}{-b} + 1 = 0$, which contradicts our assumption that there does not exists an element in $F$ such that $a^2 + a + 1 = 0$.

\eproof

c) If $F \in \{ \mathbb{Z}_3, \mathbb{Z}_7 \}$, then $F^c$ is a field and $F'$ is not a field, and if $F \in \{ \mathbb{Z}_2, \mathbb{Z}_5 \}$, then $F'$ is a field and $F^c$ is a field.

\bproof

If $F \in \{ \mathbb{Z}_3, \mathbb{Z}_7 \}$, then $\forall a \in F, \ a \cdot a + 1 \neq 0$; thus, $F^c$ is a field by part a. We can show this exhaustively by making $f: \mathbb{Z}_3 \rightarrow \mathbb{Z}_3$ such that $f(x) := x \cdot x + 1$ and $g: \mathbb{Z}_7 \rightarrow \mathbb{Z}_7$ such that $g(x) := x \cdot x + 1$.
\begin{align*}
f(0) \neq & 0 \\
f(1) \neq & 0 \\
f(2) \neq & 0 \\
g(0) \neq & 0 \\
g(1) \neq & 0 \\
g(2) \neq & 0 \\
g(3) \neq & 0 \\
g(4) \neq & 0 \\
g(5) \neq & 0 \\
g(6) \neq & 0
\end{align*}
If $F \in \{ \mathbb{Z}_3, \mathbb{Z}_7 \}$, then $\exists a \in F, \ a^2 + a + 1 = 0$, namely, $a = 1 \in \mathbb{Z}_3$ and $a = 2 \in \mathbb{Z}_7$; thus, $F'$ is not a field. \\
\bigskip
If $F \in \{ \mathbb{Z}_2, \mathbb{Z}_5 \}$, then $\forall a \in F, \ a^2 + a + 1 \neq 0$; thus, $F'$ is a field by part b. We can show this exhaustively by making $f: \mathbb{Z}_2 \rightarrow \mathbb{Z}_2$ such that $f(x) := x^2 + x + 1$ and $g: \mathbb{Z}_5 \rightarrow \mathbb{Z}_5$ such that $g(x) := x^2 + x + 1$.
\begin{align*}
f(0) \neq & 0 \\
f(1) \neq & 0 \\
g(0) \neq & 0 \\
g(1) \neq & 0 \\
g(2) \neq & 0 \\
g(3) \neq & 0 \\
g(4) \neq & 0
\end{align*}
If $F \in \{ \mathbb{Z}_2, \mathbb{Z}_5 \}$, then $\exists a \in F, \ a \cdot a + 1 = 0$, namely, $a = 1 \in \mathbb{Z}_2$ and $a = 2 \in \mathbb{Z}_5$; thus, $F^c$ is not a field.

\eproof

\newpage

4. a) 

\bproof

We will consider every other case to be true and show that they all lead to contradictions: \\
Case 1: $xy = 0$. This contradicts the fact that no non-zero elements can have a product of 0. \\
Case 2: $xy = x$. This contradicts the fact that the multiplicative identity is unique. \\
Case 3: $xy = y$. This also contradicts the fact that the multiplicative identity is unique. \\
Thus $xy = 1$ must be true.

\eproof

b)

\bproof

We will consider every other case to be true and show that they will lead to contradictions: \\
Case 1: $xx = 0$ and $yy = 0$. This contradicts the fact that no non-zero elements can have a product of 0. \\
Case 2: $xx = 1$ and $yy = 1$. This contradicts the fact that we already know that $x$ and $y$ are multiplicative inverses of each other, and multiplicative inverses are unique. \\
Case 3:  $xx = x$ and $yy = y$. This contradicts the fact that the multiplicative identity is unique.\\
Thus $xx = y$ and $yy = x$ must be true.

\eproof

c) 

\bproof

We will consider every other case to be true and show that they will lead to contradictions: \\
Case 1: $x + y = 0$. Multiplying both side by $x$ gives us the equation $xx + xy = 0$, and thus, $y + 1= 0$. If instead we multiply the original equation by $y$ gives us the equation $xy + yy = 0$, and thus, $x + 1 = 0$. Using the transitive property of the equality relation, we can say that $x+1 = y+1$. Hence, $x = y$ which is a contradiction because $x$ and $y$ are distinct. \\
Case 2: $x + y = x$. This contradicts the fact that the additive identity is unique. \\
Case 3:  $x + y = y$. This contradicts the fact that the additive identity is unique. \\
Thus $x + y = 1$ must be true.

\eproof

d)

\bproof

We will consider every other case to be true and show that they will lead to contradictions: \\
Case 1: $x + x = 1$. By the transitive property, we can say that $x + x = x + y$, and if we apply the additive inverse of $x$ to both sides, we get $x = y$, a contradiction. \\
Case 2: $x + x = x$. This contradicts the fact that additive identities are unique. \\
Case 3: $x + x = y$. If we multiply $x$ on both sides, we get $xx + xx = xy$. That leaves us with $y + y = 1$. This allows us to use the transitive property to say $y + y = x + y$, and by applying the additive inverse of $y$ we get $y = x$, a contradiction. \\
Thus $x + x = 0$ must be true, and in a similar fashion, we can show that $y + y = 1$ must be true. \\
\bigskip

Because $x$ and $y$ are their own additive inverses, $x + 1 =y$ and $y + 1 = x$ is trivially true through rearrangement. $1 + 1 = 0$ because 1 needs an additive inverse, and we have already shown it cannot be $y$ or $x$; thus, it has to be 1.

\eproof

\newpage

5. a)

\bproof

Consider the sequence $(a_n)_{n \in \mathbb{N}}$ of real numbers and notice that for $I_{n+1} \subseteq I_n$, $a_n \leq a_{n+1}$. Because $(a_n)$ is bounded above by at most $b_1$, then by definition the set of all $a_n$ has a supremum. Consider $x := Sup(\{ a_n \})$, if $\exists n \in \mathbb{N}$ such that $a_n > x$ then it contradicts $x$ being an upper bound. If $\exists \epsilon > 0, \ \forall n > N \in \mathbb{N}: \ |x-a_n| \geq \epsilon$, then it contradicts that $x$ is the lowest upper bound of $\{ a_n \}$. Thus $\forall \epsilon > 0, \ \exists n > N \in \mathbb{N}: |x-a_n| < \epsilon$. Because $(a_n)$ is non-decreasing then $\forall n > N \in \mathbb{N}, \ |x-a_n| \leq |x-a_N| < \epsilon$. Thus we can say that $(a_n)$ converges to $Sup( \{ a_n \} )$. Similarly, it is easy to prove that $(b_n)_{n \in \mathbb{N}}$ converges to $Inf( \{ b_n \} )$. Because $(a_n)$ and $(b_n)$ converge, then $ \bigcap_{n \in \mathbb{N}}I_n$ is a defined interval. We know that $Sup( \{ a_n \} ) \in \bigcap_{n \in \mathbb{N}}I_n$ and $Inf( \{ b_n \} ) \in \bigcap_{n \in \mathbb{N}}I_n$ due to the least upper bound property of $\mathbb{R}$; thus, $\bigcap_{n \in \mathbb{N}}I_n$ contains all of its boundaries and has at least 1 element, making it closed and non-empty. (This makes heavy use of the monotone convergence theorem).

\eproof

b)

\bproof

Consider $I_n = [a_n,b_n]$, where $a_1 = -10$ and $b_1 = 10$ $a_{n+1} \in \{ x | x \in \mathbb{Q} \text{ and } \pi - \frac{1}{n+1} < x < \pi \text{ and } x > a_n \}$ and $b_{n+1} \in \{ x | x \in \mathbb{Q} \text{ and } \pi < x < \pi + \frac{1}{n+1} \text{ and } x < b_n \}$ (These sets are non-empty because of 2b; thus, we can invoke the Axiom of Choice). As $n$ becomes arbitrarily large, the distance $| \pi - x | < \epsilon, \ \forall x \in I_n $. If we choose any object $x \in I_n$, then either $x>\pi$ or $x<\pi$. If we consider the case where $x > \pi$ then $\exists \epsilon > 0$ such that $x = \pi + \epsilon$. Therefore, we can find some $m>n$ such that $\pi < b_m < x$; thus, $x$ cannot be in $\bigcap_{n \in \mathbb{N}}I_n$. We can make a similar argument if $x< \pi$. Because $\pi \notin \mathbb{Q}$, $\bigcap_{n \in \mathbb{N}}I_n = \emptyset$. Thus, we have provided a counterexample.

\eproof



\end{flushleft}
\end{document}