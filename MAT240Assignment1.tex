\documentclass[11pt]{article}
\usepackage[utf8]{inputenc}
\usepackage{amssymb, amsmath, amsthm, changepage}

\title{MAT240 Assignment 1}
\author{Nicolas Coballe}

\newcommand{\bproof}{\begin{proof}
$ $ \\
\begin{adjustwidth}{3em}{0pt}
}

\newcommand{\eproof}{\end{adjustwidth}
\end{proof}}

\begin{document}

\maketitle
\begin{flushleft}

3. a)
\bproof

Consider $ g $ and $ g' $ are both inverses of $f$. \\
\begin{center}
 $ g(f(x)) = x = g'(f(x)), \ \forall x \in X $ \\
\end{center}
Since $x$ was chosen arbitrarily, $f(x)$ is just an arbitrary object in $Y$. 
Thus, $g(y) = g'(y), \ \forall y \in Y$ and $g = g'$

\eproof

b) $f:\mathbb{R} \rightarrow \mathbb{R}, \ f(x) = x^2$ is not invertible because it is not a bijection. This is because $1 \neq -1$ but $f(1) = f(-1)$ (This is shown by the problem below.)

\bigskip

c) 
\bproof
$\Rightarrow$ \\
We will first prove that $f$ is a surjection. \\
By assumption $\forall y \in Y, \ f(g(y)) = y$ and because $g(y)$ is just an arbitrary element of $X$, this shows that $\forall y \in Y, \ \exists x \in X \ \text{st} \ f(x) = y$ \newline
\\
We will now show that $f$ is an injection. \\
Assume that $f$ is not an injection. \\
Thus, $\exists x, x' \in X$ st $x \neq x'$ and $f(x) = f(x')$ \\
Then $g(f(x)) = g(f(x')) = x = x'$ and thus we have reached a contradiction. \newline
\\
Thus we have proven that $f$ is bijective. \newline
\\
$\Leftarrow$ \\
Because $f:X \rightarrow Y$ is a bijective function, then \\
$\forall y \in Y, \exists ! x \in X$ st $f(x) = y$ \\
Thus we can construct a function $g: Y \rightarrow X$ that maps all $y \in Y$ to the unique $x \in X$ st $f(x) = y$ \\
Thus, by construction $g(f(x)) = x$ and $f(g(y)) = y$

\eproof

\bigskip

d) It does not follow that $f \circ g = I_Y$ \\

\bproof
Consider the function $f:\{ a \} \rightarrow \{ 1,2 \}$ such that $f(a) = 1$ 
and the function $g:\{ 1,2 \} \rightarrow \{ a \}$ such that $g(1) = a$ and $g(2) = a$ \\
Thus $(g \circ f)(a) = a$ and $(g \circ f) = I_X$
But $(f \circ g)(2) = 1$ so $(f \circ g) \neq I_Y$ \\
Thus we have constructed a counterexample.
\eproof

e) It does follow that $f \circ g = I_Y$ now.

\bproof
If $g \circ f = I_X$ then $f$ is injective because if we assume $f$ is not injective, then we result in the contradiction where $g(f(x)) = g(f(x')) = x = x'$ when $x \neq x'$ \\
Thus $f$ is injective and surjective, making it a bijection, and by part c, this implies that it is invertible and there exists a $g$ st $f \circ g = I_Y$
\eproof


\end{flushleft}
\end{document}