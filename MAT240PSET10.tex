\documentclass[11pt]{article}
\usepackage[utf8]{inputenc}
\usepackage{amssymb, amsmath, amsthm, changepage, graphicx, caption, subcaption}

\graphicspath{{./images/}}

\title{MAT240 Problem Set 10}
\author{Nicolas}

\newcommand{\R}{\mathbb{R}}

\newcommand{\N}{\mathbb{N}}

\newcommand{\Z}{\mathbb{Z}}

\newcommand{\F}{\mathbb{F}}

\newcommand{\C}{\mathbb{C}}

\newcommand{\Q}{\mathbb{Q}}

\newenvironment{myproof}
{\begin{proof} \begin{adjustwidth}{3em}{0pt}$ $\par\nobreak\ignorespaces}
{\end{adjustwidth} \end{proof}} 

\begin{document}

\maketitle
\begin{flushleft}

\textit{Lemma 1.1} If an operator $F: \C[x] \to \C[x]$ has the specified properties of 1.1 then $F(x^n) = nx^{n-1}$.

\begin{myproof}

We will use induction on $n$. \\
Base Case: n = 1. $D(x) = 1 = 1x^0$. Thus the base case is true.
Induction Hypothesis: If $D(x^n) = nx^{n-1}$ then $D(x^{n+1}) = (n+1)x^n$.
inductive Step: $D(x^{n+1}) = D(x \cdot x^n) = D(x)x^n+ xD(x^n) = x^n + xnx^{n-1} = (n+1)x^n$ as desired.

\end{myproof}

\textit{Lemma 1.2} If an operator $F: \C[x,e] \to \C[x,e]$ has the specified properties of 1.3 then $F(e^n) = ne^n$.

\begin{myproof}

We will use induction on $n$. \\
Base Case: n = 1. $D(x) = 1 = 1x^0$. Thus the base case is true.
Induction Hypothesis: If $D(e^n) = ne^n$ then $D(e^{n+1}) = (n+1)e^{n+1}$.
inductive Step: $D(e^{n+1}) = D(e \cdot e^n) = D(e)e^n+ eD(e^n) = e^{n+1} + ne^{n+1} = (n+1)e^{n+1}$ as desired.

\end{myproof}

1. 1)

\begin{myproof}

Consider $D': \C[x] \to \C[x]$ with the mentioned properties. Consider a polynomial $f \in \C[x]$ such that $f(x) = a_0 + a_1x + a_2x^2 + \cdots + a_nx^n$. Then $D'f$ is defined as $Df(x) = D(a_0) + D(a_1x) + D(a_2x^2) + \cdots + D(a_nx^n) = a_0D(1) + a_1D(x) + a_2D(x^2) + \cdots + a_nD(x^n)$. Using \textit{Lemma 1.1} we know that this is equal to $a_1 + 2a_2x + \cdots + na_nx^{n-1}$. Thus $D'f = Df$; however since $f$ was arbitrary $D = D'$ meaning that $D$ is unique.
\end{myproof}

1. 2)

\begin{myproof}

Considering that any polynomial of degree $n, \ n \geq 1$ differentiates into a polynomial of degree $n-1$, that means that no polynomial of degree $n \geq 1$ can be an Eigenvector. Thus, only constant polynomials can be Eigenvectors. Consider the polynomial $f(x) = \lambda, \ \lambda \in \C$. Then $Df = 0$ which is a scalar multiple of $\lambda$, thus 0 is an Eigenvalue with every constant polynomial being an Eigenvector because $\lambda$ was arbitrary.

\end{myproof}

1. 3)

\begin{myproof}

Clearly, all the Eigenvectors in $U$ are still Eigenvectors in V with Eigenvalue 0. However, in $V$ we have some new Eigenvalues and Eigenvectors. Because of \textit{Lemma 1.1}, $D(e^n) = ne^n$ which is a scalar multiple of $e^n$ by $n$. meaning that any natural number is an Eigenvalue with Eigenvector being of the form $f(x,e) = e^n, \ \forall n \in \N$. As for generalized Eigenvectors, every polynomial that does not have a factor of $e$ will eventually differentiate into 0. Thus every polynomial of solely $x$ is a generalized Eigenvector of Eigenvalue 0. Consider if you take $D^m(\lambda e^n) = \lambda n^me^n$. Thus, $e^n$ is a generalized Eigenvector with Eigenvalue $n^m$ for any $n,m \in \N$.

\end{myproof}

1. 4)

\begin{myproof}

$D^3-4D^2+5D-2 = (D-2)(D-1)^2$. Thus, 1 and 2 are Eigenvalues. By performing the Jordan Basis algorithm, we find that the basis for the null space of $(D-2)$ is just $e^2$ while the basis for the null space of $(D-1)^2$ is just $e$. Thus the Jordan Basis for $S$ is $e,e^2$.

\end{myproof}

\newpage

2.

\begin{myproof}

Let $f(x,e) = e^a$. Then $(D-a)f = 0$. However for $k \in \N$. There is no other function that maps to $e^a$ from $(D-a)^k$ thus $e^a$ is the only basis vector of $W_a$.

\end{myproof}

\newpage

3.

\begin{myproof}
The Jordan Canonical Form of the matrix is as follows.
\begin{align*}
A_J = \begin{bmatrix}
1 & 0 & 0 & 0 & 0 \\
0 & 1 & 1 & 0 & 0 \\
0 & 0 & 1 & 1 & 0 \\
0 & 0 & 0 & 1 & 1 \\
0 & 0 & 0 & 0 & 1
\end{bmatrix}
\end{align*}
Let $P$ be a invertible matrix defined as follows:
\begin{align*}
P = \begin{bmatrix}
0 & -1 & 0 & 0 & 0 \\
0 & 0 & 1 & 0 & 0 \\
-1 & 0 & 0 & 1 & -1 \\
-1 & 0 & 0 & 0 & 0 \\
0 & 0 & 0 & 0 & 1
\end{bmatrix} \\
P^{-1} = \begin{bmatrix}
0 & 0 & 0 & -1 & 0 \\
-1 & 0 & 0 & 0 & 0 \\
0 & 1 & 0 & 0 & 0 \\
0 & 0 & 1 & -1 & 1 \\
0 & 0 & 0 & 0 & 1
\end{bmatrix}
\end{align*}
Notice that $P^{-1}AP = A_J$ as desired.
\end{myproof}

\newpage

4. 1)

\begin{myproof}

Consider any object in $\tau$

\end{myproof}

\end{flushleft}

\end{document}