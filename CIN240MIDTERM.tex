\documentclass[12pt]{article}
\usepackage[utf8]{inputenc}
\usepackage{amssymb, amsmath, amsthm, changepage}

\title{CIN240 MIDTERM}
\author{Nicolas Coballe}

\begin{document}

\maketitle
\begin{flushleft}
\textit{Part 1} \\
1. d) Revisionist \\
2. b) The European Art Film\\
3. c) Integrationist \\
4. 1) Emphatic film style\\
5. b) La Politique des auteurs \\
6. d) All of the above \\
7. b) Individualized \\
8. a) Psychological Causality\\
9. b) Anti-democratic Tendencies\\
10. b) The Paramount Decree 
\newpage

\textit{Part 2} \\
1. \\
Blaxploitation is a type of exploitation film meant to appeal to people by featuring a predominantly black cast and telling black stories. They were most popular during the late 60s and early 70s, and they typically had plots focused around modern life at the time. Like other exploitation films, blaxploitation films generally had lower budgets, with its main selling point being the black culture. These films frequently took place in large urban areas, where the characters were allowed to explore the big city; this represented the new-found, yet restricted, freedom and social mobility that black people gained during this time period as a result of the Civil Rights movement. As opposed to previous films with black cast, in blaxploiation films, the characters were more active, expressive, and autonomous. Blaxploitation films like, \textit{The Spook who Sat by the Door,} agreed with black insurrection in favour of black integraton, choosing non-conformity by having the main method of conflict resolution being "gratuitous" violence. This violence was not with purpose though as it highlights that the social restrictions put on black people made violence a necessity to survive in this integrated society.\\
\bigskip
2. \\



\end{flushleft}
\end{document}