\documentclass[11pt]{article}
\usepackage[utf8]{inputenc}
\usepackage{amssymb, amsmath, amsthm, changepage, graphicx, caption, subcaption}

\graphicspath{{./images/}}

\title{MAT247 Problem Set 1}
\author{Nicolas}

\newcommand{\R}{\mathbb{R}}

\newcommand{\N}{\mathbb{N}}

\newcommand{\Z}{\mathbb{Z}}

\newcommand{\F}{\mathbb{F}}

\newcommand{\C}{\mathbb{C}}

\newcommand{\Q}{\mathbb{Q}}

\newcommand{\norm}[1]{\left\lVert#1\right\rVert}

\newcommand{\inn}[2]{\langle#1,#2\rangle}

\newenvironment{myproof}
{\begin{proof} \begin{adjustwidth}{3em}{0pt}$ $\par\nobreak\ignorespaces}
{\end{adjustwidth} \end{proof}} 

\begin{document}

\maketitle
\begin{flushleft}

1.

\begin{myproof}

Consider $\inn{(0,1,0)}{(0,1,0)} = 0$; however, $(0,1,0) \neq 0$, meaning that the function is not definite.

\end{myproof}

\newpage

2.

\begin{myproof}

If there exists an $x \in V$ such that $(x,x) > 0$, then $x^2(1,1)>0$ by linearity and symmetry. Because $x^2$ is positive (if $x^2= 0$ then the inequality doesn't hold) then $(1,1)$ must also be positive. Thus for any $y \in V \setminus \{ 0 \}$, $(y,y) = y^2(1,1) > 0$. And for 0, $(0,0) = 0$. Thus, definiteness is achieved.

\end{myproof}

\newpage

3.

\begin{myproof}

$\Leftarrow$ \\
\bigskip
Let $a,b \in \R$
\begin{align*}
\norm{au+bv}^2 = & \ \norm{bu + av}^2 \\
\inn{au+bv}{au+bv} = & \ \inn{bu+av}{bu+av} \\
\inn{au}{au+bv} + \inn{bv}{au+bv} = & \ \inn{bu}{bu+av} + \inn{av}{bu+av} \\
\inn{au}{au} + \inn{au}{bv} + \inn{bv}{av} + \inn{bv}{bv} = & \ \inn{bu}{bu} +\inn{bu}{av} + \inn{av}{bu} + \inn{av}{av} \\
a \norm{u}^2 + b \norm{v}^2 = & \ b \norm{u}^2 + a \norm{v}^2
\end{align*}
but because $a,b$ were arbitrary, this holds for $a=1$ and $b=0$. Then, we have $\norm{u}^2 = \norm{v}^2 \implies \norm{u} = \norm{v}$. \\
\bigskip
$\Rightarrow$ \\
\bigskip
Let $a,b \in \R$
\begin{align*}
\norm{u} = & \ \norm{v} \\
\norm{u}^2 = & \ \norm{v}^2 \\
\inn{u}{u} = & \ \inn{v}{v} \\
\inn{au}{au} + \inn{au}{bv} + \inn{bv}{av} + \inn{bv}{bv} = & \ \inn{bu}{bu} +\inn{bu}{av} + \inn{av}{bu} + \inn{av}{av} \\
\norm{au+bv}^2 = & \ \norm{bu + av}^2 \\
\end{align*}
As desired.

\end{myproof}

\newpage

4.

\begin{myproof}

Consider the $V=\R^4$ with the the Euclidean Inner-product. Let $u = (\sqrt{a},\sqrt{b},\sqrt{c},\sqrt{d})$ and $v = (\sqrt{\frac1a} , \sqrt{\frac1b} , \sqrt{\frac1c} , \sqrt{\frac1d)}, \ a,b,c,d \in \R$. By \textit{Cauchy-Schwarz} $|\inn{u}{v}| \leq \norm{u} \norm{v}$; however, $\inn{u}{v} = 4$. Thus $4 \leq \norm{u} \norm{v}$. But because $\norm{x} = \sqrt{\inn{x}{x}}$ for all $x \in \R^4$. Thus, $4 \leq \sqrt{a + b + c + d} \sqrt{\frac1a + \frac1b + \frac1c + \frac1d} \implies 16 \leq (a+b+c+d)(\frac1a + \frac1b + \frac1c + \frac1d)$ as desired.

\end{myproof}

\end{flushleft}
\end{document}