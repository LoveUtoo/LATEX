\documentclass[11pt]{article}
\usepackage[utf8]{inputenc}
\usepackage{amssymb, amsmath, amsthm, changepage}

\title{MAT157 Problem Set 4}
\author{Nicolas Coballe}

\newcommand{\R}{\mathbb{R}}

\newcommand{\N}{\mathbb{N}}

\newcommand{\Z}{\mathbb{Z}}

\newenvironment{myproof}
{\begin{proof} \begin{adjustwidth}{3em}{0pt}$ $\par\nobreak\ignorespaces}
{\end{adjustwidth} \end{proof}} 

\begin{document}

\maketitle
\begin{flushleft}


1.1 $f:\R \rightarrow \R, \ f := \begin{cases}
x \text{ if } x < 0\\
x + 1 \text{ if } x \geq 0\\
\end{cases}$
\begin{myproof}
This function is injective because it is strictly increasing on its domain, but it is not surjective because there is no $x \in \R$ such that $f(x) = 0$.
\end{myproof}

1.2 $g:\R \rightarrow \R, \ g := \begin{cases}
x + 1 \text{ if } x < 0 \\
x \text{ if } x \geq 0\\
\end{cases}$
\begin{myproof}
This function is surjective because for any $x < 0$ there exists $x_0 := x-1$ such that $f(x_0) = x$ and for any $x \geq 0$ there exists $x_1 := x$ such that $f(x_1) = x$. It is not injective because $f(0) = 0$ and $f(-1) = 0$. 
\end{myproof}

\newpage

2.1

\begin{myproof}
Consider for the sake of contradiction that $\frac{p+q}{q}$ is not in lowest terms. Then there exists $a \neq b, n \in N$ such that $p + q = an$ and $q = bn$. Thus, we can write $p = an -bn = (a-b)n$. Thus, $\frac{p}{q} = \frac{(a-b)n}{bn}$. Thus $\frac{p}{q}$ is not in reduced form, a contradiction. \\
\bigskip
Similarly, $\frac{p}{p+q} = \frac{(a-b)n}{an}$. Thus, $\frac{p}{q}$ is not in reduced form, a contradiction.
\end{myproof}

2.2

\begin{myproof}
If $\frac{p}{q}$ is in reduced form, then $p \neq q$. Thus, either $p<q$ or $p>q$. \\
\bigskip
Case 1: $p<q$. $\frac{p}{q}$ has a parent, namely $\frac{p}{q-p}$ because $\frac{p}{(q-p)+p} = \frac{p}{q}$. This parent is unique because if $p<q$ then $\frac{p}{q}$ and its parent must have the same numerator, and there is only one number that when added to $p$ results in $q$, namely $q-p$. \\
Case 2: $p>q$. $\frac{p}{q}$ has a parent, namely $\frac{p-q}{q}$ because $\frac{(p-q)+q}{q} = \frac{p}{q}$. This parent is unique because if $p>q$ then $\frac{p}{q}$ and its parent must have the same denominator, and there is only one number that when added to $q$ results in $p$, namely $q-p$.
\end{myproof}

2.3
\begin{myproof}
The ancestry of $\frac{13}{17}$ is $\{ \frac{13}{17} , \frac{13}{4}, \frac{9}{4}, \frac{5}{4}, \frac{1}{4}, \frac{1}{4}, \frac{1}{3}, \frac{1}{2}, \frac{1}{1} \}$.
\end{myproof}

2.4

\begin{myproof}
Because every positive ration in reduced form, $\frac{p}{q}$ has a parent being either $\frac{p}{q-p}$ if $p<q$ or $\frac{p-q}{q}$ if $p>q$. Thus, in either case either the numerator or the denominator of the parent is strictly less than the child. Therefore every generation will either decrease the numerator or the denominator. If we iterate this a sufficient, finite amount of times, the numerator and denominator of each generation will decrease until $p = q =1$ because $p,q \in \N$. After $p = q = 1$, there are no further parents. Thus, every positive ration in reduced form, $\frac{p}{q}$ has a finite ancestry that contains 1.
\end{myproof}

\newpage

3.a

\begin{myproof}
Consider any $x \in (a,b)$ and choose $\epsilon := \text{ min}\{|x-a|,|x-b|\}$. Notice that this number is non-zero because $x \neq a$ or $x \neq b$.
\begin{align*}
0 \leq |x-y| < \epsilon \\
-\epsilon < x-y < \epsilon \\
-x-\epsilon < -y < -x +\epsilon \\
x + \epsilon > y > x - \epsilon
\end{align*} 
But because $\epsilon = \text{min}(|x-a|,|x-b|)$, then $y \in (x-\epsilon, x + \epsilon) \subseteq (a,b)$. Therefore, $y \in (a,b)$; thus, for every $y \in \R$ such that $|x-y| < \epsilon$, $y \in U$. \\
\bigskip
Similarly consider any $x \in \R \setminus [a,b]$ and choose $\epsilon := \text{ min}\{|x-a|,|x-b|\}$ as well. Because of our choice of $\epsilon$, $(x-\epsilon, x + \epsilon) \cap [a,b] = \emptyset$. Thus, $y \in \R \setminus [a,b]$. \\
\bigskip
$[a,b)$ is not open because if $x = a$ then for any $\epsilon > 0$ if we consider $y = a - \frac{1}{n}$  and choose $n$ arbitrarily large such that $\frac{1}{n} < \epsilon$, then $| x- y| = |a-(a-\frac{1}{n})| = |\frac{1}{n}| < \epsilon$. But $y \notin [a,b)$ because $y < a$.
\end{myproof}

3.b

\begin{myproof}
For any $x \in \bigcup_{i \in I}U_i$ then there exists $j \in I$ such that $x \in U_j$. By assumption, there exists some $\epsilon > 0$ such that for every $y \in R,$ $|x-y| < \epsilon \implies y \in U_j$. Because $\forall z \in U_j, \ z \in \bigcup_{i \in I}U_i$. Then if $|x-y| < \epsilon \implies y \in \bigcup_{i \in I}U_i$.
\end{myproof}

3.c

\begin{myproof}
We will prove this by induction on $n$. \\
\bigskip
Base Case: $n = 1, \ \bigcap_{n = 1}^{1}U_i $. This set is trivially open. \\
Induction Hypothesis: If $\bigcap_{n = 1}^{n}U_i  $ is open, then $\bigcap_{n = 1}^{n+1}U_i$ is open. \\
Inductive Step: $\bigcap_{n = 1}^{n+1}U_i = (\bigcap_{n = 1}^{n}U_i) \cap (U_{n+1})$. If $\bigcap_{n = 1}^{n+1}U_i$ and $U_{n+1}$ are disjoint, then $\bigcap_{n = 1}^{n+1}U_i = \emptyset$, making it vacuously open. Consider that $\bigcap_{n = 1}^{n+1}U_i$ and $U_{n+1}$ are not disjoint, thus we can choose an arbitrary $x \in (\bigcap_{n = 1}^{n}U_i) \cap (U_{n+1})$. Because both $\bigcap_{n = 1}^{n}U_i$ and $U_{n+1}$ are open, then $\exists \epsilon_0 > 0, \ \forall y \in \R: |x-y| < \epsilon_0 \implies y \in \bigcap_{n = 1}^{n}$ and $\exists \epsilon_1 > 0, \ \forall y \in \R: |x-y| < \epsilon_1 \implies y \in U_{n+1}$. Thus, if we choose $\epsilon := \text{ min}\{ \epsilon_0, \epsilon_1 \}$. Then $|x-y| < \epsilon \implies y \in (x+ \epsilon, x -\epsilon)$. But because of our choice of $\epsilon$, $(x- \epsilon, x +\epsilon) \subseteq \bigcap_{n = 1}^{n}U_i$ and $(x- \epsilon, x +\epsilon) \subseteq U_{n+1}$. Thus, $(x-\epsilon, x+ \epsilon) \subseteq (\bigcap_{n = 1}^{n}U_i) \cap (U_{n+1})$. Because our choice of $x$ was arbitrary, $\bigcap_{n = 1}^{n+1}U_i$ is open. \\
\bigskip
Thus the finite intersection of open set is also open.
\end{myproof}

3.d

\begin{myproof}
Consider the sequence of sets $(U_n)_{n = 1}^{\infty}$ such that $U_n := (- \frac{1}{n}, \frac{1}{n})$ and $\bigcap_{n = 1}^{\infty} U_n$. Clearly, 0 is in this set, but there exists no $\epsilon > 0$ with $y \in \R$ such that $|0-y| < \epsilon \implies y \in \bigcap_{n = 1}^{\infty} U_n$. For sake of contradiction, assume that there exists an $\epsilon >0, \forall y \in \R: |0-y| < \epsilon \implies y \in \bigcap_{n = 1}^{\infty} U_n$. We can choose $k$ arbitrarily large such that $k > \frac{3 \epsilon}{4}$. Thus, take $y = \frac{3 \epsilon}{4}$, and thus $|0-y| = \frac{3 \epsilon}{4} < \epsilon$, but $y \notin U_k$, thus $y \notin \bigcap_{n = 1}^{\infty} U_n$. This contradicts the assumption that $\epsilon >0, \forall y \in \R: |0-y| < \epsilon \implies y \in \bigcap_{n = 1}^{\infty} U_n$.
\end{myproof}

\newpage

4.

\begin{myproof}
Consider the sequence of sets $(X_i)_{i=0}^\infty$ defined recursively as $X_0 := (0,1]$ and $X_i := (\text{max}(X_{i-1}), 2^i -1]$ and the sequence of sets $(Y_i)_{i=0}^\infty$ defined recursively as $Y_0 := (-1,0)$ and $Y_i := (-2^i +1, \text{min}(Y_{i-1}))$. Then we can define a function $f: \R \setminus \{0 \}$ such that $f(x) := \begin{cases} 1-x \text{ if } x \in X_0 \\ 2^i + \text{max}(X_{i-1}) -x \text{ if } x \in X_i \\ 1-x \text{ if } x \in Y_0 \\ -2^i - \text{inf}(Y_{i-1}) -x \text{ if } x \in Y_i  \end{cases}$. \\
This function is a bijection.
\end{myproof}

\newpage

5.

\begin{myproof}
Consider the polynomial $p$ of degree $n$, and $X(p)$ be the set of roots of polynomial $p$. Each polynomial can be expressed as $(a_0, a_1, ..., a_n) \in \Z^n$. Notice that every polynomial in this set has finite roots, meaning it has countable roots. Because $\Z$ is countable and the cartesian product of countable sets is countable, then we can use $\Z^n$ as a countable indexing set. Thus, $\bigcup_{i \in \Z^n}X(p_i)$ is a countable union of countable sets, making it countable. But since $n$ was arbitrary, the set of all roots of polynomials degree $N \in \N$ are countable.
\end{myproof}


\end{flushleft}
\end{document}