\documentclass[11pt]{article}
\usepackage[utf8]{inputenc}
\usepackage{amssymb, amsmath, amsthm, changepage, graphicx, caption, subcaption}

\graphicspath{{./images/}}

\title{CSC240 Problem Set 2}
\author{Nicolas}

\newcommand{\R}{\mathbb{R}}

\newcommand{\N}{\mathbb{N}}

\newcommand{\Z}{\mathbb{Z}}

\newcommand{\F}{\mathbb{F}}

\newcommand{\C}{\mathbb{C}}

\newcommand{\Q}{\mathbb{Q}}

\newcommand{\Implies}{\mbox{ IMPLIES }}
\newcommand{\Or}{\mbox{ OR }}
\newcommand{\Andd}{\mbox{ AND }}
\newcommand{\Not}{\mbox{NOT }}
\newcommand{\Iff}{\mbox{ IFF }}
\newcommand{\True}{\mbox{T}}
\newcommand{\False}{\mbox{F}}
\newcommand{\Neither}{\mbox{N}}
\newcommand{\gor}{\mbox{ GOR }}
\newcommand{\gand}{\mbox{ GAND }}
\newcommand{\gnot}{\mbox{GNOT }}
\newcommand{\grot}{\mbox{GROT }}

\newenvironment{myproof}
{\begin{proof} \begin{adjustwidth}{3em}{0pt}$ $\par\nobreak\ignorespaces}
{\end{adjustwidth} \end{proof}} 

\begin{document}

\maketitle
\begin{flushleft}

Discussed Q1 and Q2 with Aaron Ma, Vishnu Nittoor, and Kary Ishwaran.

1.

We start with:

\begin{align*}
&  (\forall i \in \Z . g(i,n)) \Implies [(\exists x \in \Z . e(i,x)) \Andd \forall j \in \Z . (g(j,i) \Implies \forall y \in \Z . e(j,y))] \\
= & \ \exists a \in \Z . (g(a,n) \Implies [(\exists x \in \Z . e(i,x)) \Andd \forall j \in \Z . (g(j,i) \Implies \forall y \in \Z . e(j,y))] \\
= & \ \exists a \in \Z . (g(a,n) \Implies \exists x \in \Z . (e(i,x) \Andd \forall j \in \Z . (g(j,i) \Implies \forall y \in \Z  . e(j,y) ) \\
= & \ \exists a \in \Z . (g(a,n) \Implies \exists x \in \Z . (e(i,x) \Andd \forall j \in \Z . \forall y \in \Z . (g(j,i) \Implies e(j,y) ) \\
= & \ \exists a \in \Z . (g(a,n) \Implies \exists x \in \Z . \forall j \in \Z . (e(i,x) \Andd \forall y \in \Z . (g(j,i) \Implies e(j,y) ) \\
= & \ \exists a \in \Z . (g(a,n) \Implies \exists x \in \Z . \forall j \in \Z . \forall y \in \Z .  (e(i,x) \Andd (g(j,i) \Implies e(j,y) ) \\
= & \ \exists a \in \Z . \exists x \in \Z . (g(a,n) \Implies \forall j \in \Z . \forall y \in \Z .  (e(i,x) \Andd (g(j,i) \Implies e(j,y) ) \\
= & \ \exists a \in \Z . \exists x \in \Z . \forall j \in \Z . (g(a,n) \Implies \forall y \in \Z .  (e(i,x) \Andd (g(j,i) \Implies e(j,y) ) \\
= & \ \exists a \in \Z . \exists x \in \Z . \forall j \in \Z . \forall y \in \Z . (g(a,n) \Implies  (e(i,x) \Andd (g(j,i) \Implies e(j,y) )
\end{align*}

Steps: \\
1. $(\forall x \in D. p(x)) \Implies E$ is transformed into $\exists x \in D. (p(x) \Implies E)$. \\
2. $(\exists x \in D. p(x)) \Andd E$ is transformed into $\exists x \in D. (p(x) \Andd E)$. \\
3. $(E \Implies (\forall x \in D. q(x)$ is transformed into $\forall x \in D. (E \Implies q(x))$. \\
4. $E \Andd (\forall x \in D. p(x))$ is transformed into $\forall x \in D. (E \Andd p(x))$. \\
5. Same step as 4. \\
6. $E \Implies (\exists x \in D. q(x))$ is transformed into $\exists x \in D. (E \Implies q(x))$. \\
7. $E \Implies (\forall x \in D. q(x))$ is transformed into $\forall x \in D. (E \Implies q(x))$. \\
8. Same step as 7.


\newpage

2. a)

\begin{myproof}
For all $i, \ 1 \leq i \leq 6$ define $f_i: \{ T, F, N \} \to \{ T, F, N \}$

\begin{align*}
f_1(P) := P, \ & f_1(T) = T, \ f_1(F) = F, \ f_1(N) = N. \\
f_2(P) := \gnot P, \ & f_2(T) = F, \ f_2(F) = T, \ f_2(N) = N. \\
f_3(P) := \grot P, \ & f_3(T) = N, \ f_3(F) = T, \ f_3(N) = F. \\
f_4(P) := \grot (\grot P), \ & f_4(T) = F, \ f_4(F) = N, \ f_4(N) = T. \\
f_5(P) := \gnot (\grot P), \ & f_5(T) = N, \ f_5(F) = F, \ f_5(N) = T. \\
f_5(P) := \gnot (\grot P), \ & f_6(T) = N, \ f_6(F) = N, \ f_6(N) = F. \\
\end{align*}
We know these are all of these bijective functions because there are $n!$ bijections between a set of $n$ elements and itself, and this set has 3 elements.

\end{myproof}

b)

\begin{myproof}

Let us define $\mathcal{F},\mathcal{T},\mathcal{N}: \{T,F,N\} \to \{T,F,N\}$ \\ $\mathcal{F}(P) = (\grot P \gand \grot (\grot P)) \gand \grot (\grot (\grot P))$ \\ $\mathcal{T}(P) = (\grot P \gor \grot (\grot P)) \gor \grot (\grot (\grot P))$ \\ 
$\mathcal{N}(P) = \grot (T(P)) \gand \grot (T(P))$ \\
\bigskip
Notice that $\forall P \in \{ T, F, N \}, \ \mathcal{T}(P) = T, \ \mathcal{F}(P) = F, \ \mathcal{N}(P) = N$ as desired.

\end{myproof}

c)

\begin{myproof}

Consider a function $f:\{T,F,N\} \to \{T,F,N\}$. Consider $f(T) = x_0, \ f(F) = x_1, \ f(N) = x_2, \ x_0,x_1,x_2 \in \{ T, F, N \}$. Then $f(P) = (( \grot (\mathcal{N}(P) \gor P)) \gand (\mathcal{T} \gand x_0))$\\ $\gor (( \grot (\mathcal{N}(P) \gor \grot (P))) \gand (\mathcal{T} \gand x_1))$ \\ $ \gor (( \grot (\mathcal{N}(P) \gor \grot (\grot(P)))) \gand (\mathcal{T} \gand x_2))$. \\
\bigskip
Because $x_0,x_1,x_2$ were arbitrary, any arbitrary function is expressible. Notice that we are using $( \grot (\mathcal{N}(P) \gor P))$ as a sort of indicator function.

\end{myproof}

d)

\begin{myproof}

$\grot$ is not expressible using only $\gand, \gor$, and $\gnot$. Because there is no univariable expression of compositions of $\gand, \gor$, $\gnot$ such that $f(N) = F$.

\end{myproof}

e)

\begin{myproof}

Because $\{T,F,N \} \times \{T, F, N\}$ has 9 elements and $\{ T, F, N \}$ has 3 elements, we can easily conclude combinatorially, that there are $3^9$ different functions from $\{T,F,N \} \times \{T, F, N\} \to \{ T,F,N \}$.

\end{myproof}

f)

\begin{myproof}

For any arbitrary function $\{T,F,N \} \times \{T, F, N\}$, we can similarly define a new indicator function defined to be the $\gand$ of the indicators between each coordinate in $\{ T,F,N \}$.

\end{myproof}


\end{flushleft}

\end{document}