\documentclass[11pt]{article}
\usepackage[utf8]{inputenc}
\usepackage{amssymb, amsmath, amsthm, changepage, graphicx, caption, subcaption}

\graphicspath{{./images/}}

\title{MAT157 Problem Set 8}
\author{Nicolas Coballe}

\newcommand{\R}{\mathbb{R}}

\newcommand{\N}{\mathbb{N}}

\newcommand{\Z}{\mathbb{Z}}

\newcommand{\F}{\mathbb{F}}

\newcommand{\C}{\mathbb{C}}

\newcommand{\Q}{\mathbb{Q}}

\newenvironment{myproof}
{\begin{proof} \begin{adjustwidth}{3em}{0pt}$ $\par\nobreak\ignorespaces}
{\end{adjustwidth} \end{proof}} 

\begin{document}

\maketitle
\begin{flushleft}

1.

\begin{myproof}

If we take $X = \begin{bmatrix}
2 & \frac{1}{3} \\
1 & \frac{2}{3}
\end{bmatrix}$ with our original transformation $A = \begin{bmatrix}
-2 & 1 \\
1 & -2
\end{bmatrix}$ then $AX = \begin{bmatrix}
-3 & 0 \\
0 & -1
\end{bmatrix}$ thus, we now know that $\frac{dx_1}{dt} = -3x_1$ and $\frac{dx_2}{dt} = -1x_2$. Thus, $e^{-3}t = ax_1 + bx_2$ and $e^{-t}=cx_1 + dx_2$. Therefore, $x_1 = \frac{e^{-3t}+e^{-t}-x_2(b+d)}{a+c}$ and $x_1 = \frac{e^{-3t}+e^{-t}-x_1(a+c)}{b+d}$

\end{myproof}

\newpage

2.

\begin{myproof}

Consider our basis for $\R^2$ is the standard basis. $A(1,0) = (\frac{2}{3},\frac{1}{3})$ and $A(0,1) = (\frac{1}{2},\frac{1}{2})$. Thus we can write the matrix of this transformation as:

\begin{align*}
A=
\begin{bmatrix}
\frac{2}{3} & \frac{1}{2} \\
\frac{1}{3} & \frac{1}{2}
\end{bmatrix}
\end{align*}
With this, $A(\frac{3}{2},1) = (\frac{3}{2},1)$ and $A(-1,1) = (\frac{-1}{6},\frac{1}{6})$, thus our Eigenvalues are 1 and $\frac{1}{6}$. These are the only Eigenvalues because the dimension of $\R^2$ is 2. Thus we can write the diagonal matrix as such:
\begin{align*}
D=
\begin{bmatrix}
1 & 0 \\
0 & \frac{1}{6}
\end{bmatrix}
\end{align*}

We can also make our change of basis matrices as such:

\begin{align*}
P=
\begin{bmatrix}
\frac{3}{2} & -1 \\
1 & 1
\end{bmatrix} \\
P^{-1}=
\begin{bmatrix}
\frac{2}{5} & \frac{2}{5} \\
\frac{-2}{5} & \frac{3}{5}
\end{bmatrix}
\end{align*}

And as one can verify $PDP^{-1} = A$ Now using the fact that $A^n = PD^nP^{-1}$ we can simply calculate $A^n$ by abusing the diagonal matrix $D$ because $D^n$ has indices $d^n_{ij} = (d_{ij})^n$. Thus:
\begin{align*}
A^n = \begin{bmatrix}
\frac{3}{2} & -1 \\
1 & 1
\end{bmatrix}
\begin{bmatrix}
1^n & 0 \\
0 & \frac{1}{6}^n
\end{bmatrix}
\begin{bmatrix}
\frac{2}{5} & \frac{2}{5} \\
\frac{-2}{5} & \frac{3}{5}
\end{bmatrix}
\end{align*}

Thus:
\begin{align*}
A(\frac{1}{2}, \frac{1}{2}) = & \ (\frac{7}{12}, \frac{5}{12}) \approx (0.5833,0.4166) \\
A^2(\frac{1}{2}, \frac{1}{2}) = & \ (\frac{43}{72}, \frac{29}{72}) \approx (0.5833,0.4027)\\
A^3(\frac{1}{2}, \frac{1}{2}) = & \ (\frac{259}{432}, \frac{173}{432}) \approx (0.5995,0.4027)\\
\end{align*}
Thus, we will handwavily say this converges to $(0.6, 0.4)$ because we don't know what our metric is.
\end{myproof}

\newpage

3.

\begin{myproof}

Every $\lambda \in \F$ is an Eigenvalue because if we consider the vector $(x_0, \lambda x_0, \lambda^2 x_0,...)$ for any $x_0 \in \F$. Then $T(x_0, \lambda x_0, \lambda^2 x_0,...) =  (\lambda x_0, \lambda^2 x_0, \lambda^3 x_0,...) = \lambda(x_0, \lambda x_0, \lambda^2 x_0,...)$. Thus, the Eigenvectors of these Eigenvalues are precisely the vectors of the form $(x_0, \lambda x_0, \lambda^2 x_0,...), \ \forall x_0, \lambda \in \F$.

\end{myproof}

\newpage

4.

\begin{myproof}

Because we know $V$ is finite dimensional, we also know that all linear operators of $V$ have a diagonal matrix consisting of its Eigenvalues. Thus, we know that if we multiply two diagonal matrices, $S = [s_{ij}]$ and $T = [t_{ij}]$. Then $ST = [st_{ij}]$ has $st_{ii} = s_{ii}t_{ii} = t_{ii}s_{ii} = ts_{ii}$ for all $1 \leq i \leq \text{dim } V$. And since on non-diagonal coordinates, the indexes are all zero. Since $ST$ and $TS$ are also diagonal matrices, then their diagonals are the Eigenvalues which are the same. This falls apart for infinite-dimensional vector spaces because we do not know if all linear operators have a diagonal matrix.

\end{myproof}


\end{flushleft}

\end{document}