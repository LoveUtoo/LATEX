\documentclass[11pt]{article}
\usepackage[utf8]{inputenc}
\usepackage{amssymb, amsmath, amsthm, changepage, graphicx, caption, subcaption}

\graphicspath{{./images/}}

\title{MAT240 Problem Set 9}
\author{Nicolas Coballe}

\newcommand{\R}{\mathbb{R}}

\newcommand{\N}{\mathbb{N}}

\newcommand{\Z}{\mathbb{Z}}

\newcommand{\F}{\mathbb{F}}

\newcommand{\C}{\mathbb{C}}

\newcommand{\Q}{\mathbb{Q}}

\newenvironment{myproof}
{\begin{proof} \begin{adjustwidth}{3em}{0pt}$ $\par\nobreak\ignorespaces}
{\end{adjustwidth} \end{proof}} 

\begin{document}

\maketitle
\begin{flushleft}

1. a)

\begin{myproof}



\end{myproof}

\newpage

3. 1)

\begin{myproof}

Consider two stochastic matrices $A=[a_{ij}]$ and $B=[b_{ij}]$ in $\F^{n \times n}$ with the property that $a_{1j} + a_{2j} + \cdots + a_{nj} = 1$ and $b_{1j} + b_{2j} + \cdots + b_{nj} = 1, \ \forall j, \ 1 \leq j \leq n$ and $a_{ij}, b_{ij} \in \R^+$. If we take the product of $A$ and $B$, $C = [c_{ij}]$ such that:
\begin{align*}
[c_{ij}] = a_{i1}b_{1j} + a_{i2}b_{2j} + \cdots + a_{in}b_{nj}
\end{align*}
Thus, if we take the sum of the $j$-th columns of $C$, we get:
\begin{align*}
c_{1j} + c_{2j} + \cdots + c_{nj} = & \ \sum_{k = 1}^n a_{1k}b_{kj} + \sum_{k = 1}^n a_{2k}b_{kj} + \cdots + \sum_{k = 1}^n a_{nk}b_{kj} \\
= & \ b_{1j} \sum_{k = 1}^n a_{kj} + b_{2j} \sum_{k = 1}^n a_{kj} + \cdots + b_{nj} \sum_{k = 1}^n a_{kj}
\end{align*}
However, we know that $\sum_{k = 1}^n a_{kj} = 1, \ \forall j, 1 \leq j \leq n$. Thus:
\begin{align*}
= & \ b_{1j} + b_{2j} + \cdots b_{nj} \\
= & \ 1
\end{align*}
Because we chose the column arbitrarily, each column in $C$ adds to 1. Because each entry in $C$ is a sum of non-negative real numbers (product of two non-negatives is non-negative), each entry in $C$ is also non-negative. Thus, the product of two stochastic matrices is indeed another stochastic matrix.


\end{myproof}

3. 2)

\begin{myproof}

\newpage

4.

\begin{myproof}

Let $P$ be a pentagon consisting of vertices $\{1,2,3,4,5\}$. Then:
\begin{align*}
P_\Gamma = \begin{bmatrix}
0 & 1 & 0 & 0 & 1 \\
1 & 0 & 1 & 0 & 0 \\
0 & 1 & 0 & 1 & 0 \\
0 & 0 & 1 & 0 & 1 \\
1 & 0 & 0 & 1 & 0
\end{bmatrix}
\end{align*}
This matrix has a Eigenvalues $2$, $-\frac{1 + \sqrt{5}}{2}$ and $\frac{1+\sqrt{5}}{2}$. This is because $P(1,1,1,1,1) = (2,2,2,2,2)$, $P(-1, \frac{1 + \sqrt{5}}{2}, -\frac{1 + \sqrt{5}}{2}, 1, 0) = (\frac{1 + \sqrt{5}}{2}, -(\frac{1 + \sqrt{5}}{2})^2, (\frac{1 + \sqrt{5}}{2})^2, -(\frac{1 + \sqrt{5}}{2}), 0)$, and $P(-1, -\frac{1 + \sqrt{5}}{2}, \frac{1 + \sqrt{5}}{2}, 1, 0) = (-\frac{1 + \sqrt{5}}{2}, (\frac{1 + \sqrt{5}}{2})^2, -(\frac{1 + \sqrt{5}}{2})^2, (\frac{1 + \sqrt{5}}{2}), 0)$. We know that theses are all the Eigenvalues of because the null space of $A - \lambda I$ has dimension 3.

\end{myproof}

\end{myproof}

\newpage

5. 

\begin{myproof}

Consider the map $A: \F^X \oplus \F^Y \to F^{X \sqcup Y}, \ A(f,g) = h, \ h(x) = \begin{cases} f(x) \text{ if } x \in X \\ g(x) \text{ if } x \in Y \end{cases}$. This is indeed a map because $X$ and $Y$ are disjoint. This map is surjective because if we consider any function $h: X \sqcup Y \to \F$, there exists an element in $F^X \oplus F^Y, \ (f,g)$ such that if $\forall x \in X, \ h(x) = f(x)$ and if $\forall x \in Y, \ h(x) = g(x)$. This map is injective because consider $A(f,g) = A(f',g')$. Then, $\forall x \in X \sqcup Y, \ A(f,g)(x) = A(f',g')(x)$. If $x \in X$, then $A(f,g)(x) = f(x) = A(f',g')(x) = f'(x)$. Similarly, if $x \in Y$, then $A(f,g)(x) = g(x) = A(f',g')(x) = g'(x)$. Thus, if $A(f,g) = A(f',g')$, then $f=f'$ and $g=g'$. Thus we have shown that there exists a isomorphism between $\F^X \oplus \F^Y$ and $F^{X \sqcup Y}$.
\end{myproof}


\end{flushleft}

\end{document}