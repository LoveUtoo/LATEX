\documentclass[11pt]{article}
\usepackage[utf8]{inputenc}
\usepackage{amssymb, amsmath, amsthm, changepage}

\title{MAT240 Problem Set 3}
\author{Nicolas Coballe}

\newcommand{\bproof}{\begin{proof}
$ $ \\
\begin{adjustwidth}{3em}{0pt}
}

\newcommand{\eproof}{\end{adjustwidth}
\end{proof}}

\begin{document}

\maketitle
\begin{flushleft}

1.

\bproof

We will now show that $\mathbb{R}^2$ with the operators $\tilde{+}$ and $\tilde{\cdot}$ define a vector space over $\mathbb{R}$. \\
\bigskip

\textbf{Commutativity} \\
\bigskip

Consider $x, y \in \mathbb{R}^2$ with $x = (a, b)$ and $y = (u, v), \ a,b,u,v \in \mathbb{R}$. Then:
\begin{align*}
x \tilde{+} y = & (a + u + 1, y + v - 1) \\
= & (u + a + 1, v + y -1) \\
= & y \tilde{+} x
\end{align*}

Thus, vector addition is commutative. \\
\bigskip

\textbf{Associativity} \\
\bigskip

Consider $x, y ,z \in \mathbb{R}^2$ with $x = (a, b), \ y = (u , v)$, and $z = (\phi, \psi), \ a,b,u,v,\phi,\psi \in \mathbb{R}$. Then:
\begin{align*}
x \tilde{+} ( y \tilde{+} z) = & (a, b) \tilde{+} (u + \phi + 1, v + \psi -1) \\
= & (a + (u + \phi +1) +1, b + (v + \psi -1) -1) \\
= & ((a + u + 1) + \phi + 1, (b + v -1) + \psi - 1) \\
= & (a + u + 1, b + v -1) \tilde{+} (\phi, \psi) \\
= & (x \tilde{+} y) \tilde{+} z
\end{align*}
Now consider $\lambda, \iota \in \mathbb{R}$. Then:
\begin{align*}
(\lambda \iota) \tilde{\cdot} x = & ( \lambda \iota a + \lambda \iota - 1, \lambda \iota b - \lambda \iota + 1) \\
= & (\lambda \iota a + \lambda \iota - \lambda + \lambda -1, \lambda \iota b - \lambda \iota - \lambda + \lambda -1 ) \\
= & (\lambda (\iota a + \iota -1) + \lambda - 1, \lambda (\iota b - \iota -1) - \lambda + 1) \\
= & \lambda \tilde{\cdot} ( \iota a + \iota -1, \iota b - \iota + 1) \\
= & \lambda \tilde{\cdot} (\iota \tilde{\cdot} x)
\end{align*}
Thus, vector addition and scalar multiplication is associative. \\
\bigskip

\textbf{Additive Identity} \\
\bigskip
There exists an element $0 \in \mathbb{R}^2$, namely $0 = (-1,  1)$, such that for all $x \in \mathbb{R}^2, \ x \tilde{+} 0 = x$. Consider $x = (a, b), \ a,b \in \mathbb{R}$. Then:
\begin{align*}
x \tilde{+} 0 = & (a -1 + 1, b + 1 -1) \\
= & (a, b) \\
= & x
\end{align*}

\textbf{Additive Inverse} \\
\bigskip
For all $x \in \mathbb{R}^2, \ x = (a,b), \ a,b \in \mathbb{R}$, there exists an additive inverse, namely $-x = (-a - 2, -b + 2)$, such that $x \tilde{+} -x = 0$. Then:
\begin{align*}
x \tilde{+} -x = & (a - a -2 + 1, b - b + 2 - 1) \\
= & (-2 + 1, 2 - 1) \\
= & (-1 , 1) \\
= & 0
\end{align*}

\textbf{Multiplicative Identity} \\
\bigskip
There exists an element $1 \in \mathbb{R}$ such that for all $x \in \mathbb{R}, \ 1 \tilde{\cdot} x = x$. Consider $x = (a, b), \ a,b \in \mathbb{R}$. Then:
\begin{align*}
1 \tilde{\cdot} x = & (1a + 1 - 1, 1b - 1 + 1) \\
= & (a, b) \\
= & x
\end{align*}

\textbf{Distributive Property} \\
\bigskip
Consider $x, y \in \mathbb{R}^2$ with $x = (a, b)$ and $y = (u, v), \ a,b,u,v \in \mathbb{R}$ and $\lambda \in \mathbb{R}$. Then:
\begin{align*}
\lambda \tilde{\cdot} (x \tilde{+} y) = & \lambda \tilde{\cdot} (a + u + 1, b + v - 1) \\
= & (\lambda (a + u + 1) + \lambda - 1, \lambda (b + v -1) - \lambda + 1) \\
= & (\lambda a + \lambda u + \lambda + \lambda - 1, \lambda b + \lambda v - \lambda - \lambda + 1) \\
= & (\lambda a + \lambda - 1 + \lambda u + \lambda - 1 + 1, \lambda b - \lambda + 1 + \lambda v - \lambda + 1 - 1)\\
= & (\lambda a + \lambda - 1, \lambda b - \lambda + 1) \tilde{+} (\lambda u + \lambda - 1, \lambda v - \lambda + 1) \\
= & \lambda \tilde{\cdot} x \tilde{+} \lambda \tilde{\cdot} y
\end{align*}
Now consider $\iota \in \mathbb{R}$. Then:
\begin{align*}
(\lambda + \iota) \tilde{\cdot} x = & ((\lambda + \iota)a + (\lambda + \iota) -1, (\lambda + \iota) b - (\lambda + \iota) + 1) \\
= & ( \lambda a + \iota a + \lambda + \iota -1,  \lambda b + \iota b - \lambda - \iota +1 ) \\
= & (\lambda a + \lambda -1 + \iota a + \iota - 1 + 1, \lambda b - \lambda +1 + \iota b - \iota b + 1) \\
= & (\lambda a + \lambda - 1, \lambda b - \lambda + 1) \tilde{+} (\iota a + \iota -1, \iota b - \iota + 1) \\
= & \lambda \tilde{\cdot} x \tilde{+} \iota \tilde{\cdot} x
\end{align*}
Thus, vector addition and scalar multiplication are linked through the distributive property. \\
\bigskip
Therefore, $\mathbb{R}^2$ with the operators $\tilde{+}$ and $\tilde{\cdot}$ define a vector space over $\mathbb{R}$.

\eproof

\newpage

2. a)

\bproof
These are all of the linear subspaces of $(\mathbb{F}_5)^2$:
\begin{align*}
&\{ (x,y): x,y \in \mathbb{F}_5 \} \\
&\{ (x, 0): x \in \mathbb{F}_5 \} \\
&\{ (0, x): x \in \mathbb{F}_5 \} \\
&\{ (x,x): x \in \mathbb{F}_5 \} \\
&\{ (x,2x): x \in \mathbb{F}_5 \} \\
&\{ (x,3x): x \in \mathbb{F}_5 \} \\
&\{ (x,4x): x \in \mathbb{F}_5 \} \\
&\{ (0,0) \}
\end{align*}
There are 8.

\eproof

b)

\bproof
These are all of the linear subspaces of $(\mathbb{F}_2)^3$:
\begin{align*}
&\{ (x,y,z): x,y,z \in \mathbb{F}_2 \} \\
&\{ (x,y,0): x,y \in \mathbb{F}_2 \} \\
&\{ (x,0,y): x,y \in \mathbb{F}_2 \} \\
&\{ (0,x,y): x,y \in \mathbb{F}_2 \} \\
&\{ (x,0,0): x \in \mathbb{F}_2 \} \\
&\{ (0,x,0): x \in \mathbb{F}_2 \} \\
&\{ (0,0,x): x \in \mathbb{F}_2 \} \\
&\{ (x,0,x): x \in \mathbb{F}_2 \} \\
&\{ (x,y,x): x \in \mathbb{F}_2 \} \\
&\{ (x,x,0): x \in \mathbb{F}_2 \} \\
&\{ (0,x,x): x \in \mathbb{F}_2 \} \\
&\{ (x,x,y): x,y \in \mathbb{F}_2 \} \\
&\{ (x,y,y): x,y \in \mathbb{F}_2 \} \\
&\{ (x,x,x): x \in \mathbb{F}_2 \} \\
&\{ (x, y, x+y): x,y \in \mathbb{F}_2 \} \\
&\{ (0,0,0) \}
\end{align*}
There are 16.

\eproof

\newpage

3. a)

\bproof

We will use $(\mathbb{F}_5)^2_T$ to denote the set of affine linear subspaces of $(\mathbb{F}_5)^2$ modelled on $T$. \\
\bigskip
$T_1 = \{ (x,y): x,y \in \mathbb{F}_5 \}$ \\
$(\mathbb{F}_5)^2_{T_1} = \{ (0,0) \}$ \\
$|(\mathbb{F}_5)^2_{T_1}| = 1$ \\
All elements are equivalent because $T = (\mathbb{F}_5)^2$. Thus, for all $x,y \in \mathbb{F}_5, \ y -x \in T_1$. \\
\bigskip
$T_2 = \{ (x, 0): x \in \mathbb{F}_5 \}$ \\
$(\mathbb{F}_5)^2_{T_2} = \{ (0,0), (0,1), (0,2), (0,3), (0,4) \}$ \\
$|(\mathbb{F}_5)^2_{T_2}| = 5$ \\
For any two elements $x=(u, v)$ and $y=(z, v), \ u,v,z \in \mathbb{F}_5$, $y - x = (s, 0), \ s \in \mathbb{F}_5$. Since $s$ is arbitrary, $(s, 0) \in T_2$; thus every element with an equivalent second coordinate are equivalent. \\
\bigskip
$T_3 = \{ (0, x): x \in \mathbb{F}_5 \}$ \\
$(\mathbb{F}_5)^2_{T_3} = \{ (0,0), (1,0), (2,0), (3,0), (4,0) \}$ \\
$|(\mathbb{F}_5)^2_{T_3}| = 5$ \\
Symmetrically to the previous affine linear subspace, every element with an equivalent first coordinate are equivalent. \\
\bigskip
$T_4 = \{ (x,x): x \in \mathbb{F}_5 \}$ \\
$(\mathbb{F}_5)^2_{T_4} = \{ (0,0), (0,1), (0,2), (0,3), (0,4) \}$ \\
$|(\mathbb{F}_5)^2_{T_4}| = 5$ \\
Consider $x = (0, b)$ and $y = (a, a + b), \ a,b \in \mathbb{F}_5$, then $y - x = (a,a) \in T_4$. Thus, $\{ (0,b): b \in \mathbb{F}_5 \}$ are the set of equivalence classes because all elements in $(\mathbb{F}_5)^2$ can be written in the form $(a, a +b), \ a,b \in \mathbb{F}_5$. \\
\bigskip
$T_5 = \{ (x,2x): x \in \mathbb{F}_5 \}$ \\
$(\mathbb{F}_5)^2_{T_5} = \{ (0,0), (0,1), (0,2), (0,3), (0,4) \}$ \\
$|(\mathbb{F}_5)^2_{T_5}| = 5$ \\
Consider $y = (a, b), \ a,b \in \mathbb{F}_5$. Since $T_5 = \{ (0,0), (1,2), (2,4), (3,1), (4,3) \}$ we can choose $x = (0, c)$ such that $b - c = 0$ if $a = 0$, $b - c = 2$ if $a = 1$, $b - c = 4$ if $a = 2$, $b - c = 1$ if $a = 3$, and $b - c = 3$ if $a = 4$. Thus, if we choose $x$ correctly then $y - x \in T_5$; thus, $\{ (0,b): b \in \mathbb{F}_5 \}$ defines the set of equivalence classes. \\
\bigskip
$T_6 = \{ (x,3x): x \in \mathbb{F}_5 \}$ \\
$(\mathbb{F}_5)^2_{T_6} = \{ (0,0), (0,1), (0,2), (0,3), (0,4) \}$ \\
$|(\mathbb{F}_5)^2_{T_6}| = 5$ \\
Consider $y = (a, b), \ a,b \in \mathbb{F}_5$. Since $T_6 = \{ (0,0), (1,3), (2,1), (3,4), (4,2) \}$ we can choose $x = (0, c)$ such that $b - c = 0$ if $a = 0$, $b - c = 3$ if $a = 1$, $b - c = 1$ if $a = 2$, $b - c = 4$ if $a = 3$, and $b - c = 2$ if $a = 4$. Thus, if we choose $x$ correctly then $y - x \in T_5$; thus, $\{ (0,b): b \in \mathbb{F}_5 \}$ defines the set of equivalence classes. \\
\bigskip
$T_7 = \{ (x,4x): x \in \mathbb{F}_5 \}$ \\
$(\mathbb{F}_5)^2_{T_7} = \{ (0,0), (0,1), (0,2), (0,3), (0,4) \}$ \\
$|(\mathbb{F}_5)^2_{T_7}| = 5$ \\
Consider $y = (a, b), \ a,b \in \mathbb{F}_5$. Since $T_6 = \{ (0,0), (1,4), (2,3), (3,2), (4,1) \}$ we can choose $x = (0, c)$ such that $b - c = 0$ if $a = 0$, $b - c = 3$ if $a = 1$, $b - c = 1$ if $a = 2$, $b - c = 4$ if $a = 3$, and $b - c = 2$ if $a = 4$. Thus, if we choose $x$ correctly then $y - x \in T_5$; thus, $\{ (0,b): b \in \mathbb{F}_5 \}$ defines the set of equivalence classes. \\
\bigskip
$T_8 = \{ (0,0) \}$ \\
$(\mathbb{F}_5)^2_{T_8} = \{ (x,y): x,y \in \mathbb{F}_5 \}$ \\
$|(\mathbb{F}_5)^2_{T_8}| = 25$ \\
This because the only elements that have a difference equal to the zero vector, are identical elements. \\
\bigskip
Thus, there are $\sum_{n=1}^8 |(\mathbb{F}_5)^2_{T_n}| = 56$ total affine linear subspaces of $(\mathbb{F}_5)^2$ modelled on $(T_n)_{n=1}^8$.

\eproof

b)

\bproof

We will use $(\mathbb{F}_2)^3_T$ to denote the set of affine linear subspaces of $(\mathbb{F}_2)^3$ modelled on $T$. \\
\bigskip
$T_1 = \{ (x,y,z): x,y,z \in \mathbb{F}_2 \}$ \\
$(\mathbb{F}_2)^3_{T_1} = \{ (0,0) \}$ \\
$|(\mathbb{F}_5)^2_{T_1}| = 1$ \\
All elements are equivalent because $T = (\mathbb{F}_5)^2$. Thus, for all $x,y \in \mathbb{F}_5, \ y -x \in T_1$. \\
\bigskip
$T_2 = \{ (x,y,0): x,y \in \mathbb{F}_2 \}$ \\
$(\mathbb{F}_2)^3_{T_2} = \{ (0,0,0), (0,0,1) \}$ \\
$|(\mathbb{F}_5)^2_{T_2}| = 2$ \\
There is a dichotomy between vectors; either the last coordinate is 1 or the last coordinate is 0. If you take the difference between any two vector with last coordinate 0, you get a vector with last coordinate being 0; thus, being in the set. If you take the difference between any two vectors with last coordinate 1, you get a vector with last coordinate being 0; thus, being in the set as well. \\
\bigskip
$T_3 = \{ (x,0,y): x,y \in \mathbb{F}_2 \}$ \\
$(\mathbb{F}_2)^3_{T_3} = \{ (0,0,0), (0,1,0) \}$ \\
$|(\mathbb{F}_5)^2_{T_3}| = 2$ \\
Again we can take the difference between any two vectors with second coordinate 0, and get a vector with second coordinate 0. We can similarly do the same to any two vectors with second coordinate 1, creating a vector with second coordinate 0. \\
\bigskip
$T_4 = \{ (0,x,y): x,y \in \mathbb{F}_2 \}$ \\
$(\mathbb{F}_2)^3_{T_4} = \{ (0,0,0), (0,1,0) \}$ \\
$|(\mathbb{F}_5)^2_{T_4}| = 2$ \\$T_16 = \{ (0,0,0) \}$ \\
Again we can take the difference between any two vectors with first coordinate 0, and get a vector with second coordinate 0. We can similarly do the same to any two vectors with second coordinate 1, creating a vector with second coordinate 0. \\
\bigskip
$T_5 = \{ (x,0,0): x \in \mathbb{F}_2 \}$ \\
$(\mathbb{F}_2)^3_{T_5} = \{ (0,0,0), (0,1,0), (0,0,1), (1,1,1) \}$ \\
$|(\mathbb{F}_5)^2_{T_5}| = 4$ \\
$(0,0,0)$ and $(1,0,0)$ are trivially equal. $(1,1,0) - (0,1,0) = (1, 0, 0)$, $(0,1,1) - (1,1,1) = (1,0,0)$, and $(1,0,1)-(0,0,1) = (1,0,0)$. \\
\bigskip
$T_6 = \{ (0,x,0): x \in \mathbb{F}_2 \}$ \\
$(\mathbb{F}_2)^3_{T_6} = \{ (0,0,0), (0,1,0), (0,0,1), (1,1,1) \}$ \\
$|(\mathbb{F}_5)^2_{T_6}| = 4$ \\
$(0,0,0)$ and $(0,1,0)$ are trivially equal. $(1,1,0) - (1,0,0) = (0, 1, 0)$, $(0,1,1) - (0,0,1) = (0,1,0)$, and $(1,0,1)-(1,1,1) = (0,1,0)$. \\
\bigskip
$T_7 = \{ (0,0,x): x \in \mathbb{F}_2 \}$ \\
$(\mathbb{F}_2)^3_{T_7} = \{ (0,0,0), (0,1,0), (0,0,1), (1,1,1) \}$ \\
$|(\mathbb{F}_5)^2_{T_7}| = 4$ \\
$(0,0,0)$ and $(0,0,1)$ are trivially equal. $(1,0,1) - (1,0,0) = (0, 0, 1)$, $(0,1,1) - (0,1,0) = (0,0,1)$, and $(1,1,1)-(1,1,0) = (0,0,1)$. \\
\bigskip
$T_8 = \{ (x,0,x): x \in \mathbb{F}_2 \}$ \\
$(\mathbb{F}_2)^3_{T_8} = \{ (0,0,0), (0,1,0), (0,0,1), (1,1,1) \}$ \\
$|(\mathbb{F}_5)^2_{T_8}| = 4$ \\
$(0,0,0)$ and $(1,0,1)$ are trivially equal. $(1,1,1) - (0,1,0) = (1, 0, 1)$, $(0,1,1) - (1,1,0) = (1,0,1)$, and $(1,0,0)-(0,0,1) = (1,0,1)$. \\
\bigskip
$T_9 = \{ (x,y,x): x,y \in \mathbb{F}_2 \}$ \\
$(\mathbb{F}_2)^3_{T_9} = \{ (0,0,0), (1,1,0), (1,0,0) \}$ \\
$|(\mathbb{F}_5)^2_{T_9}| = 3$ \\
$(0,0,0)$, $(1,0,1)$, $(0,1,0)$, and $(1,0,1)$ are trivially equal. $(1,0,0) - (0,0,1) = (1, 0, 1)$ and $(0,1,1)-(1,1,0) = (1,0,1)$. \\
\bigskip
$T_{10} = \{ (x,x,0): x \in \mathbb{F}_2 \}$ \\
$(\mathbb{F}_2)^3_{T_{10}} = \{ (0,0,0), (1,0,0), (1,0,1), (1,1,1) \}$ \\
$|(\mathbb{F}_5)^2_{T_{10}}| = 4$ \\
$(0,0,0)$ and $(1,1,0)$ are trivially equal. $(1,0,0) - (0,1,0) = (1, 1, 0)$, $(1,1,1)-(0,0,1) = (1,1,0)$, and $(0,1,1)-(1,0,1) = (1,1,0)$. \\
\bigskip
$T_{11} = \{ (0,x,x): x \in \mathbb{F}_2 \}$ \\
$(\mathbb{F}_2)^3_{T_{11}} = \{ (0,0,0), (1,1,1), (1,1,0), (0,0,1) \}$ \\
$|(\mathbb{F}_5)^2_{T_{11}}| = 4$ \\
$(0,0,0)$ and $(0,1,1)$ are trivially equal. $(1,1,1) - (1,0,0) = (0, 1, 1)$, $(1,1,0)-(1,0,1) = (0,1,1)$ and $(0,0,1)-(0,1,0) = (0,1,1)$. \\
\bigskip
$T_{12} = \{ (x,x,y): x,y \in \mathbb{F}_2 \}$ \\
$(\mathbb{F}_2)^3_{T_{12}} = \{ (0,0,0), (1,0,1), (1,0,0) \}$ \\
$|(\mathbb{F}_5)^2_{T_{12}}| = 3$ \\
$(0,0,0)$, $(1,1,1)$, $(1,1,0)$ and $(0,0,1)$ are trivially equal. $(1,0,1) - (0,1,0) = (1, 1, 1)$ and $(1,0,0)-(0,1,1) = (1,1,1)$. \\
\bigskip
$T_{13} = \{ (x,y,y): x,y \in \mathbb{F}_2 \}$ \\
$(\mathbb{F}_2)^3_{T_{13}} = \{ (0,0,0), (1,1,0), (0,1,0) \}$ \\
$|(\mathbb{F}_5)^2_{T_{13}}| = 3$ \\
$(0,0,0)$, $(1,1,1)$, $(1,0,0)$ and $(0,1,1)$ are trivially equal. $(1,1,1) - (0,0,1) = (1, 1, 1)$ and $(0,1,0)-(0,1,0) = (1,1,1)$. \\
\bigskip
$T_{14} = \{ (x,x,x): x,y \in \mathbb{F}_2 \}$ \\
$(\mathbb{F}_2)^3_{T_{14}} = \{ (0,0,0), (1,1,1), (1,1,0), (0,0,1) \}$ \\
$|(\mathbb{F}_5)^2_{T_{14}}| = 4$ \\
$(0,0,0)$ and $(1,1,1)$ are trivially equal. $(1,0,1) - (0,1,0) = (1, 1, 1)$, $(1,1,0)-(0,0,1) = (1,1,1)$ and $(1,0,0)-(0,1,1) = (1,1,1)$. \\
\bigskip
$T_{15} = \{ (x,y,x+y): x,y \in \mathbb{F}_2 \}$ \\
$(\mathbb{F}_2)^3_{T_{15}} = \{ (0,0,0), (1,1,1), (0,1,0) \}$ \\
$|(\mathbb{F}_5)^2_{T_{15}}| = 3$ \\
$(0,0,0)$, $(1,0,1)$, $(1,1,0)$ and $(0,1,1)$ are trivially equal. $(1,1,1) - (0,0,1) = (1, 1, 0)$ and $(1,0,0)-(0,1,0) = (1,1,0)$. \\
\bigskip
$T_16 = \{ (0,0,0) \}$ \\
$(\mathbb{F}_2)^3_{T_16} = \{ (x,y,z): x,y,z \in \mathbb{F}_2 \}$ \\
$|(\mathbb{F}_2)^3_{T_16}| = 8$ \\
This because the only elements that have a difference equal to the zero vector, are identical elements. \\
\bigskip
Thus, there are $\sum_{n=1}^16 |(\mathbb{F}_2)^3_{T_n}| = 59$ total affine linear subspaces of $(\mathbb{F}_2)^3$ modelled on $(T_n)_{n=1}^{16}$.
\eproof

\newpage

4. a)

\bproof

We will first show that the sum is a direct sum. Suppose that there exists $f \in V_e \setminus \{0\}$ and $g \in V_o \setminus \{0\}$ such that $f + g = 0$. Thus, $f = -g$, then $f(x) = f(-x) = -g(-x) = -g(x)$ This is a contradiction because if $g$ is odd and non-zero, than $-g(-x) = g(x) \neq -g(x)$ . Therefore, there is only one way to express the zero-vector in terms of vectors from $V_e$ and $V_o$, namely the sum of the zero vectors in each respective subspace. Thus, according to \textbf{1.44} in \textit{Axler's Linear Algebra Done Right}, $V_e + V_o$ is a direct sum \\
\bigskip
We will now show that any vector in $\mathbb{R}^{\mathbb{R}}$ can be expressed as a sum of even and odd functions. Consider $f \in \mathbb{R}^{\mathbb{R}}$. Consider the function $g: \mathbb{R} \rightarrow \mathbb{R}, \ g(x) = \frac{f(x) + f(-x)}{2}$. Notice that $g(-x) = \frac{f(-x)+f(x)}{2} = \frac{f(x)+f(-x)}{2} = g(x)$. Thus, $g$ is even. Consider the function $h: \mathbb{R} \rightarrow \mathbb{R}, \ h(x) = \frac{f(x) - f(-x)}{2}$. Notice that $h(-x) = \frac{f(-x)-f(x)}{2} = \frac{-f(x) + f(-x)}{2} = -h(x)$. Thus, $h$ is odd. Take the sum of $g + h = \frac{f(x)+f(-x)}{2} + \frac{f(x)-f(-x)}{2} = \frac{2f(x)}{2} = f(x) = f$.\\
\bigskip
Thus, we have shown that $V = V_e \oplus V_o$.

\eproof

b)

\bproof

$\exp(x) = \frac{\exp(x)+\exp(-x)}{2} + \frac{\exp(x)-\exp(-x)}{2}$. 

\eproof


\end{flushleft}
\end{document}