\documentclass[11pt]{article}
\usepackage[utf8]{inputenc}
\usepackage{amssymb, amsmath, amsthm, changepage, graphicx, caption, subcaption}

\graphicspath{{./images/}}

\title{MAT344 Problem Set 2}
\author{Nicolas}

\newcommand{\R}{\mathbb{R}}

\newcommand{\N}{\mathbb{N}}

\newcommand{\Z}{\mathbb{Z}}

\newcommand{\F}{\mathbb{F}}

\newcommand{\C}{\mathbb{C}}

\newcommand{\Q}{\mathbb{Q}}

\newcommand{\norm}[1]{\left\lVert#1\right\rVert}

\newcommand{\inn}[2]{\langle#1,#2\rangle}

\newenvironment{myproof}
{\begin{proof} \begin{adjustwidth}{3em}{0pt}$ $\par\nobreak\ignorespaces}
{\end{adjustwidth} \end{proof}} 

\begin{document}

\maketitle
\begin{flushleft}

1.

\begin{myproof}
The first is false if $n = 3$ because $f_4 = 13 \neq 2 f_3 - f_0 = 14$. The third is false if $n = 0$ because we can assume all indices are non-negative. $f_1 = 2 \neq 2^1 - (0-1)2^{-2} = 2 + \frac{1}{4}$. \\
\bigskip
We will prove the second statement using a combinatorial argument. Consider all the strings of length $n-2$ that do not contain $``000"$. We can concatenate, $001$ to the beginning of every string and get a string of size $n+1$ that also does not contain $``000"$. We can do the same thing for string of length $n-1$ that do not contain $``000"$. We can concatenate $01$ to the beginning of the string and get a string of size $n+1$ that also does not contain $``000"$. Lastly, consider all the strings of length $n$ that do not contain $``000"$. We can concatenate $001$ to the beginning of the string and get a string of size $n+1$ that also does not contain $``000"$. However, we know that the strings of length $n+1$ generated by the strings of length $n-2, n-1,$ and $n$ are disjoint because for the strings generated by $n-2$ The all start with $001$, while all strings generated by $n-1$ start with $01$, and all the strings generated by $n$ start with $1$. This is the only way to make strings of length $n+1$ (by concatenating to the beginning of the strings) without over-counting because concatenating anything else to the first part of our strings will lead to two of the same strings being potentially counted twice. Thus all strings of length $n-2, n-1$, and $n$ after our concatenation is applied will be in the set of strings of length $n+1$. Similarly, if we take any string of length $n+1$ with the property that it does not contain $``000"$ we have 3 cases:\\
\bigskip
Case 1: Our string starts with $1$ thus, we can anti-concatenate the $1$ and get a string of length $n$ that is in the set of strings of length $n$ that do not contain $``000"$. \\
\bigskip
Case 2: Our string starts with $01$ thus, we can anti-concatenate the $01$ and get a string of length $n$ that is in the set of strings of length $n-1$ that do not contain $``000"$. \\
\bigskip
Case 3: Our string starts with $001$ thus, we can anti-concatenate the $001$ and get a string of length $n$ that is in the set of strings of length $n-2$ that do not contain $``000"$. \\
\bigskip
Those are all the cases because any other case would contain $``000"$. Thus, we have shown how to uniquely obtain every string of length $n+1$ through concatenation of strings of length $n-2, n-1$, and $n$, and we shown the reverse by anti-concatenation is also unique. Thus, there is a bijection between the set that the left side counts and the set that the right side counts; thus, both sides must count the same number of objects.
\end{myproof}

\newpage

2. a)

\begin{myproof}
It is trivial that a total number of strings of length $n$ comprising of the elements from $[5]$ is $5^n$. Let us denote the set of all these strings as $S_n$. Consider any string, $s \in S_n$. If this string does not include $'1'$ then it has an even amount of 1s because 0 is even. Notice that ever string in $S_n$ either includes 1 or it doesn't. We know that the subsets that $E_n$ and $O_n$ count are disjoint because there cannot be a string with both an even amount of 1s and an odd amount of 1s. However, we know that the union of these two disjoints subsets is equal to $S_n$ which we know has $5^n$ elements. Thus, $E_n + O_n = 5^n$ as desired. 
\end{myproof}

b)

\begin{myproof}
Consider the set of all strings of length $n$, denoted $S_n$. Now consider $E(S_n)$ and $O(S_n)$ denoting the subsets of $S$ containing all strings with an even count of 1s and an odd count of 1s respectively. Consider we take all the strings in $E(S_n)$ and concatenated $x \in [5]$ to them. If we concatenate $'1'$ then, we get a string of length $n+1$ which has an odd amount of 1s. If we concatenate anything else, then we get a string of length $n+1$ which still has an even amount of 1s. Consider we did the same thing to $S(O_n)$. If we concatenated $'1'$ then, we get a string of length $n+1$ which has an even amount of 1s. Otherwise, we would get a string of length $n+1$ which still has an even amount of 1s. Because $S(E_n) \cup S(O_n) = S_n$, then concatenating each distinct character to ever string of length $n$ gives us the set of strings of length $n+1$ due to the recursive construction of finite strings. Thus we know that $S(E_{n+1})$ is equal to $\{ x \cdot e | \ \forall x \in \{ 2,3,4,5, \}, \forall y \in S(E_n) \} \cup \{ 1 \cdot o | \ \forall o \in S(O_n) \}$ where $\cdot$ is the concatenation (on the left) binary operation. Similarly $S(O_{n+1}$ is equal to $\{ x \cdot o | \ \forall x \in \{ 2,3,4,5, \}, \forall y \in S(O_n) \} \cup \{ 1 \cdot e | \ \forall o \in S(E_n) \}$. Thus, we have shown that $S(E_{n+1})$ contains exactly the same number of elements elements as $S(O_n)$ and 4 times the elements of $S(E_n)$ because we concatenated 4 distinct characters, and $S(O_{n+1})$ contains exactly the same number of elements as $S(O_n)$ and 4 times $S(E_n)$. Therefore, $E_{n+1} = 4E_n + O_n$ and $O_{n+1} = 4O_n + E_n$ as desired.
\end{myproof}

c)

\begin{myproof}
We will prove this using strong induction on $n$. We will also prove that $O_n = \frac{1}{2}[5^n-3^n]$. \\
\bigskip
Base Case: $n = 1$ $E_1 = 4 = \frac{1}{2}[5^1+3^1]$ and $O_1 = 1 = \frac{1}{2}[5^1-3^1]$. Thus, the base case is true. \\
\bigskip
Induction Hypothesis: If $j < n+1$ then $E_n = \frac{1}{2}[5^j+3^j]$ and $O_n = \frac{1}{2}[5^j-3^j]$. \\
\bigskip
Inductive Step:
\begin{align*}
E_{n+1} = & \ 4 E_n + O_n & \ \text{by part b)} \\
= & \ 2[5^n+3^n] + \frac{1}{2}[5^n-3^n] & \ \text{by induction hypothesis} \\
= & \ \frac{5}{2}5^n + \frac{3}{2}3^n & \\
= & \frac{1}{2}[5^{n+1} + 3^{n+1}] &
\end{align*}
as desired.
\end{myproof}

\newpage

3. a)

\begin{myproof}
We can first conclude that trivially, $a_0 = 1$,$a_1 = 0$, and $a_2 = 7$. Consider that we generate the set of all sequences of $n$ blocks given that we have the set of all $n-2$ and $n-3$ sequences. Take the set of sequences of $n-3$ which we know has $a_{n-3}$ elements. We can add 1 of 6 different blocks to the $n-3$ length sequences. Thus, from $n-3$ length sequences, we get $6 a_{n-3}$ sequences of length $n$. We can do the same thing for all sequences of length $n-2$, by simply adjoining 1 of 7 different blocks to the $n-2$ length sequences. Thus $a_n = 7 a_{n-2} + 6 a_{n-3}$
\end{myproof}

b)

\begin{myproof}
We will prove this using strong induction on $n$. \\
\bigskip
Base Cases: $n = 0$ There is 1 sequence, so this is true. $n = 1$ There are 7 sequences so this is true. $n=2$ There are 6 sequences, so this is true (you can just trust me or you can plug the numbers in yourself). \\
Induction Hypothesis: If $j < n$ then $a_j = 3^{j+2} + (-2)^{j+4} + 5(-1)^{j+1}$ \\
Inductive Step:
\begin{align*}
a_n = & \ 7a_{n-2} + 6a_{n-3} \text{ By Induction Hypothesis} \ \\
= & \frac{7(3^{n} + (-2)^{n+2} + 5(-1)^{n-1})+6(3^{n-1} + (-2)^{n+1} + 5(-1)^{n-2})}{20}  \\
= & \frac{7 \cdot 3^{n} + 7 \cdot (-2)^{n+2} + 35(-1)^{n-1})+6 \cdot 3^{n-1} + 6 \cdot (-2)^{n+1} + 30(-1)^{n-2}}{20} \\
= & \frac{7 \cdot 3^{n} + 7 \cdot (-2)^{n+2} + 35(-1)^{n+1})+2 \cdot 3^{n} + (-3) \cdot (-2)^{n+2}  -30(-1)^{n+1}}{20} \\
= & \frac{3^n(9)+(-2)^{n+2}(4)+(-1)^{n+1}(5)}{20} \\
= & \frac{3^{n+1}+(-2)^{n+4}+5(-1)^{n+1}}{20}
\end{align*}
As desired.
\end{myproof}

c)

\begin{myproof}
We will claim that $b_n = 21 b_{n-2} + 20 b_{n-3}$ with $b_n = \frac{5^{n+2}+2(-4)^{n+2}+3(-1)^{n+1}}{54}$ Consider our starting values for $b_1 = 0$, $b_2 = 21$ and $b_3 = 20$. We will prove this through strong induction on $n$. \\
\bigskip
Base Case: The statement is true for $n \in \{ 1,2,3 \}$. \\
\bigskip
Induction Hypothesis: If $j < n$ then $b_j = \frac{5^{j+2}+2(-4)^{j+2}+3(-1)^{j+1}}{54}$. \\
Inductive Step:
\begin{align*}
b_n = & 21 b_{n-2} + 20 b_{n-3}  \text{ By Induction Hypothesis} \\
= & \frac{21(5^{n}+2(-4)^{n}+3(-1)^{n}) + 20 (5^{n-1}+2(-4)^{n-1}+3(-1)^{n-2}) }{54}  \\
= & \frac{21 \cdot 5^n + 42(-4)^n + 63(-1)^n+20 \cdot 5^{n-1} +40(-4)^{n-1} + 60(-1)^{n-2}}{54} \\
= & \frac{21 \cdot 5^n + 42(-4)^n + 63(-1)^n+4 \cdot 5^{n} -10(-4)^{n} - 60(-1)^{n-1}}{54} \\
= & \frac{(25)5^n + (32)(-4)^n+(3)(-1)^{n-1}}{54} \\
= & \frac{5^{n+2} + 2(-4)^{n+1} + 3(-1)^{n-1}}{54}
\end{align*}
As desired.
\end{myproof}

\newpage

4. a)

\begin{myproof}
We will first count the number of total distinct hands there are. We will consider two cases: either the two cards dealt are the same suit or they are not.\\
\bigskip
Case 1: They are the same suit, thus there are ${13 \choose 2} = 78$ different hands of the same suit (if they're the same suit, they cannot be the same rank). \\
\bigskip
Case 2: They are not the same suit, thus we have to use combinations with replacement because the players can be dealt 2 of the same ranked cards. Thus there are $\frac{(13 + 2 -1)!}{2!(13-1)!} = 91$. \\
\bigskip
Thus in total there are 163 distinct hands. We can multiply this by 9 for each distinct position. That means in total there are 1467 distinct hands and positions; however, since there total 1557 players, by the \textit{pigeonhole principle}, there will be two players with the same hand and position.
\end{myproof}

b)

\begin{myproof}
We will simply fix an arbitrary position, $p$, and show that within the set of all players of position $p$ there must be two players in position $p$ with the same hand. Well we know that there are 1557 players in total, so we can assume that all the positions at all the tables are filled, and each position has the same number of players ($\frac{1557}{9} = 173)$. Because there are only 163 distinct hands, and we have 173 players per position, by the \textit{pigeonhole principle}, there must be two players with the same hand in position $p$. Because $p$ was arbitrary, this holds for all positions in $\{1,2,3,4,5,6,7,8,9\}$.
\end{myproof}

c)

\begin{myproof}
To calculate the number of distinct hands that are identical (rank and suit) then we simply have to take ${52 \choose 2} = 1326$. Thus, with this new equivalence relation, there are 1326 different hands, which is less than 1557, so by the \textit{pigeonhole principle}, there must be two players with the identical hands
\end{myproof}

\newpage

\textbf{Theorem}: \textit{If there are $pigeon$ pigeons and $pigeon + hole$ pigeon holes, and we put each pigeon in a hole, then there exists atleast $2\frac{hole}{hole}$ pigeons in the same pigeon hole.} $pigeon, hole \in \N$

\begin{myproof}
Since $pigeon, hole \in \N$ that means that $pigeon + hole > pigeon$, which means we can use the \textit{pigeonhole principle} to deduce that there must be at least one hole, with $2 \frac{hole}{hole} = 2$ pigeons as desired.
\end{myproof}


\end{flushleft}
\end{document}