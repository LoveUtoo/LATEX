\documentclass[11pt]{article}
\usepackage[utf8]{inputenc}
\usepackage{amssymb, amsmath, amsthm, changepage, graphicx, caption, subcaption}

\graphicspath{{./images/}}

\title{MAT344 Problem Set 1}
\author{Nicolas}

\newcommand{\R}{\mathbb{R}}

\newcommand{\N}{\mathbb{N}}

\newcommand{\Z}{\mathbb{Z}}

\newcommand{\F}{\mathbb{F}}

\newcommand{\C}{\mathbb{C}}

\newcommand{\Q}{\mathbb{Q}}

\newenvironment{myproof}
{\begin{proof} \begin{adjustwidth}{3em}{0pt}$ $\par\nobreak\ignorespaces}
{\end{adjustwidth} \end{proof}} 

\begin{document}

\maketitle
\begin{flushleft}

1.

\begin{myproof}

The left side counts how many subsets of size 2 we can make in a set of $m+n$ elements, which is $\frac{(m+n)(m+n-1)}{2}$. The right side counts how many subsets of size 2 you can make with a set of $m$ elements, with $n$ elements, and with the product of $m$ and $n$: $\frac{m(m-1)}{2} + \frac{n(n-1)}{2} + mn$. These count the same object because consider $M$ and $N$ are disjoint finite sets of size $m$ and $n$, respectfully. The left side counts the 2 element subsets of $M \cup N$, while the right side counts the 2 elements subsets of $M$ and $N$ then counts the remainder of subsets from $M \cup N$ of the form $\{a,b\}, a \in M, b  \in N$. Any $\{a,b\} \subseteq M \cup N$ must also be a subset of either $M$, $N$, or pairs of $M$ and $N$, and it is easy to see that the converse also holds. We can also see through simple algebraic manipulation, we can show the second formula is equivalent to the first.

\end{myproof}

\newpage

2.

\begin{myproof}

Notice that both sides are only non-zero iff $n\geq m > k$. The left side counts the amount of permutations of $[n]$ divided by the amount of permutations of $[n-k]$ all multiplied by the permutations of $[(n-k)]$ divided by the permutations of $[(m-k)]$. The right side counts the permutations of $[n]$ divided by the permutations $[m]$ all multiplied by the amount of permutations of $[m]$ divided by the amount of permutations of $[m-k]$. In short:
\begin{align*}
P(n,k) {n-k \choose m-k} = & \ {n \choose m} P(m,k) \\
\frac{n!}{(n-k)!}\frac{(n-k)!}{(n-m)!(m-k)!} = & \ \frac{n!}{(n-m)!m!} \frac{m!}{(m-k)!} \\
\frac{n!}{(n-m)!(m-k)!} = & \ \frac{n!}{(n-m)!(m-k)!}
\end{align*}
as desired

\end{myproof}

\newpage

3.

\begin{myproof}

First we can define up functions as functions $f:[k] \to [n]$ such that $f$ is strictly increasing and injective and we can define down functions as functions $g:[k] \to [n]$ such that $g$ is strictly decreasing and injective. We can define a concatenation of up and down functions to make an updown function. Consider $M$ is the set of functions $f:[k] \to [n]$, $N$ is the set of functions $g:[m] \to [p]$, and $K$ is the set of functions $h:[k+m] \to [n+p]$. Then $\oplus: M \times N \to K, \  \oplus(f,g)(x) \to h(x) \begin{cases} f(x) \text{ if $x \leq k$} \\ g(x) \text{ if $x > k$} \end{cases}$. Notice that any up-down function can be decomposed into a concatenation of an up function and a down function (when $k \geq 2$). Thus we can count the amount of up-down functions with domain of cardinality $n$ by taking the sum of all concatenations of $f$ and $g$ such that the domain of $\oplus(f,g)$ has a cardinality of $n$. Thus the number of up-down functions from $[k] \to [n]$ is equal to $\sum_{k=1}^n {k \choose k}{n \choose n-k}$ 

\end{myproof}


\newpage

4.

\begin{myproof}

Considering that there are $2^n$ ways to partition a set if $n$ elements into $k$ parts. Thus, we can work with our partition recursively. Therefore a partition of $n$ elements into 3 parts is the sum of every collection of $k<n-k$ elements multiplied by the number of partitions of $n-k$ into 2 parts. Thus, we can define a function $P:\N \times \N \to \N, \ P(n,m) = \sum_{k=1}^{n-k}{n \choose k} P(n-k,m-1)$. Because we have a base case where $n= 2$ we can construct $P$ recursively to give us the amount of ways we can partition a set of $n$ elements into $k$ distinct parts.

\end{myproof}

\newpage

5.

\begin{myproof}

Consider the finite subset of naturals, $m$ less than $10^n$ where $1 \leq x_0 + \cdots + x_{n-1} \leq 10$ where $x_0,...,x_{n-1}$ are digits of $m$. We can do this quite simply because this value is equal to $10^n-p(n-1)$ because we can increase each digit by one up to $9$ (where $p(n-1)$ is the amount of natural numbers whose digits sum to 10 and are less than $10^{n-1}$). The minus 10 is take away the numbers such that $\exists i, \  1 \leq i \leq n, \  x_i > 9$ or $\forall i, \ 1 \leq i \leq n, \ x_i < 1$. If the sum of numbers with $n$ digits is greater than 1 and less than 10, there exists exactly one $\lambda \in \{ 0,1,...,9 \}$ such that $\lambda + x_0 + \cdots x_{n-1} = 10$ while $\lambda 10^{n+1} + m < 10^{n+2}$ for some natural $m$ with $n$ digits having a sum less than equal to 10. Thus the amount of natural less than $10^n$ with digits summing to 10 is $10^{n-1}-p(n-2)$.

\end{myproof}

\newpage

6.

Consider the 26 character alphabet of capital English characters. Out of all the $n$-character strings, how many contain the substring ``NICOLAS" where $n \geq 7$.

\begin{myproof}

Because the substring ``NICOLAS" has to be connected, we can treat it as a singular character and it can be placed anywhere in the string. The other characters are irrelevant, so they can be any of the 26 characters we have in our alphabet. Thus, the total amount of $n$-character strings with ``NICOLAS" as a substring are $26^{n-7}$ as desired.

\end{myproof}

\end{flushleft}

\end{document}