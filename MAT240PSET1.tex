\documentclass[11pt]{article}
\usepackage[utf8]{inputenc}
\usepackage{amssymb, amsmath, amsthm, changepage}

\title{MAT240 Assignment 1}
\author{Nicolas Coballe}

\newcommand{\bproof}{\begin{proof}
$ $ \\
\begin{adjustwidth}{3em}{0pt}
}

\newcommand{\eproof}{\end{adjustwidth}
\end{proof}}

\begin{document}

\maketitle
\begin{flushleft}

1. a) $|Map(X_n, X_n)| = n^n$ if $n > 0$ or $|Map(X_n, X_n)| = 1$ if $n=0$.

\bproof

Consider the finite, non-empty set $X_n = \{ x_1, x_2, ..., x_n \}$ consisting of $n$ elements and the map $f: X_n \rightarrow X_n$. 
$\forall x_i \in X_n, \exists f(x_i) \in X_n $, and because $X_n$ has a cardinality of $n$, there are only
$n$ different values for $f(x_i)$. Then, there is a choice of $n$, $n$ different times; thus, there are $\underbrace{n \cdot n \cdots n }_\text{$n$ times}$ different possible maps $f: X_n \rightarrow X_n$. Thus the cardinality of $Map(X_n, X_n)$ is $\underbrace{n \cdot n \cdots n }_\text{$n$ times}$, or simply $n^n$.

Considering the case where $n = 0$, $X_n$ is empty, and thus we can easily say that there is one trivial map between the empty set and the empty set, so $|Map(\emptyset , \emptyset )| = 1$.

\eproof

b) $|Bij(X_n, X_n)| = n!$

\bproof

Consider the finite, non-empty set $X_n = \{ x_1, x_2, ..., x_n \}$ consisting of $n$ elements and the bijection $f: X_n \rightarrow X_n$. Suppose $f(x_1) \in X_n$, because $f$ is bijective $f(x_2) \in X_n \setminus \{ f(x_1) \} $. Thus if we continue this process for $n$ iterations we get: $\forall x_i, i>1, x_i \in X_n \setminus \{ f(x_{i-1}) \} \setminus \cdots \setminus \{ f(x_1) \}$ \\
Thus, there is a choice of $(n-i+1)$, $n$ different times ($i$ ranges from $1$ to $n$). This leaves us with $n \cdot (n-1) \cdots 2 \cdot 1$ different possible bijections $f: X_n \rightarrow X_n$. Therefore, the cardinality of $Bij(X_n, X_n)$ is $n \cdot (n-1) \cdots 2 \cdot 1$, or more concisely, $n!$! (This last ``!'' is an exclamation mark and not a second factorial.)

\bigskip
Taking into the consideration when $n = 0$ and $X_n = \emptyset$, there is only one bijection, the identity function. Thus  $Bij(\emptyset , \emptyset ) = 1 = 0!$

\eproof

\newpage

2. a)

\bproof

We will prove that composition of functions is associative:
\begin{align*}
\forall x \in S, ((h \circ g) \circ f )(x) = & (h \circ g )(f(x)) \\
= & h(g(f(x))) \\
= &h((g \circ f) (x)) \\
= & (h \circ (g \circ f))(x)
\end{align*}
And thus, $(h \circ g) \circ f = h \circ (g \circ f)$

\eproof

b) We will show all the bracketing of the compositions of 4 functions and prove that they are equivalent, and then we will show how many ways you can bracket 5 functions.

\bproof

Let us fix any $x \in S$.
\begin{align*}
(((k \circ h) \circ g ) \circ f)(x) = & ((k \circ h \circ g) \circ f)(x) \\
= & (k \circ h \circ g)(f(x)) \\
= & k(h(g(f(x)))) \\
((k \circ (h \circ g )) \circ f)(x) = & ((k \circ h \circ g) \circ f)(x) \\
= & (k \circ h \circ g)(f(x)) \\
= & k(h(g(f(x)))) \\
(k \circ ((h \circ g) \circ f))(x) = & (k \circ (h \circ g \circ f))(x) \\
= & k((h \circ g \circ f)(x)) \\
= & k(h(g(f(x))))\\
(k \circ (h \circ (g \circ f)))(x) = & (k \circ (h \circ g \circ f))(x) \\
= & k((h \circ g \circ f)(x)) \\
= & k(h(g(f(x)))) \\
((k \circ h) \circ (g \circ f))(x) = & (k \circ h) ((g \circ f)(x)) \\
= & k((h \circ g \circ f)(x)) \\
= & k(h(g(f(x)))) \\
\end{align*} \\
Thus all bracketing of the composition of 4 functions all agree with each other.

\eproof

\bproof

Consider a new function $j:W \rightarrow X$ and the composition of the 5 functions $f,g,h,k,j$, with all their different bracketing.
\begin{align*}
(((j \circ k) \circ h) \circ g) \circ f, \ & ((j \circ (k \circ h)) \circ g) \circ f, \ (j \circ ((k \circ h) \circ g)) \circ f \\
(j \circ (k \circ (h \circ g))) \circ f, \ & ((j \circ k) \circ (h \circ g)) \circ f, \ j \circ (((k \circ h) \circ g) \circ f) \\
j \circ ((k \circ (h \circ g)) \circ f), \ & j \circ (k \circ ((h \circ g) \circ f)), \ j \circ (k \circ (h \circ (g \circ f))) \\
j \circ ((k \circ h) \circ (g \circ f)), & (j \circ k) \circ ((h \circ g) \circ f), (j \circ k) \circ (h \circ (g \circ f)) \\
((j \circ k) \circ h) \circ (g \circ f), & ((j \circ k) \circ h) \circ (g \circ f)
\end{align*} 
Thus there are 14 ways to bracket 5 functions.

\eproof

c) We will prove this using induction.

\bproof

Base Case: $n = 3$, which is already proven from part a. \\
\bigskip
Induction Step: $k = n + 1$. We will prove this through cases of how to bracket $n+1$ function compositions.
If $f:X_0 \rightarrow X_{n+1}$, then because the composition of functions is a binary operation, either we have: \\
\bigskip
Case 1: $f = f_{n+1} \circ f_c $, where $f_c:X_0 \rightarrow X_n$ is a composition of $n$ functions with arbitrary bracketing. Because compositions of $n$ functions is well-defined without bracketing by the induction hypothesis, we can simply write $f_c = f_n \circ f_{n-1} \circ \cdots \circ f_1$. Thus:
\begin{align*}
\forall x \in X_0, \ (f_{n+1} \circ (f_n \circ f_{n-1} \circ \cdots \circ f_1))(x) = & f_{n+1} (f_n \circ f_{n-1} \circ \cdots \circ f_1)(x)) \\
= & f_{n-1}(f_n (f_{n-1} ... f_1(x)))
\end{align*} \\
\bigskip
Case 2: $f = f_c \circ f_1 $, where $f_c:X_1 \rightarrow X_{n+1}$ is a composition of $n$ functions with arbitrary bracketing. Because compositions of $n$ functions is well-defined without bracketing by the induction hypothesis, we can simply write $f_c = f_{n+1} \circ f_{n} \circ \cdots \circ f_2$. Thus:
\begin{align*}
\forall x \in X_0, \ ((f_{n+1} \circ f_{n} \circ \cdots \circ f_2) \circ f_1 )(x) = & (f_{n+1} \circ f_{n} \circ \cdots \circ f_2)(f_1(x)) \\
= & f_{n-1}(f_n (f_{n-1} ... f_1(x)))
\end{align*} \\
\bigskip
Case 3: There exists way bracketing such that $f = f_a  \circ f_b$, where $f_a:X_b \rightarrow X_{n+1}$ and $f_b:X_0 \rightarrow X_b$ are arbitrary composition of less than $n$ functions. Because $f_a$ and $f_b$ are of compositions of less than $n$ functions, it is also well defined without bracketing by the induction hypothesis (strong induction). Thus without confusion we can substitute $f_a=f_{n+1} \circ f_a'$, where $f_a':X_b \rightarrow X_n$. Thus, $f = (f_{n+1} \circ f_a') \circ f_b$. By the associativity of the composition of 3 functions we can rewrite each as $f = f_{n+1} \circ (f_a' \circ f_b)$. Since $(f_a' \circ f_b):X_0 \rightarrow X_n$ is an arbitrary composition of less than $n$ functions, it reduces this to just case 1.
\eproof

\newpage

3. a)
\bproof

Consider $ g $ and $ g' $ are both inverses of $f$. \\
\begin{center}
 $\forall x \in X, \ g(f(x)) = x = g'(f(x))$ \\
\end{center}
Since $x$ was chosen arbitrarily, $f(x)$ is just an arbitrary object in $Y$. 
Thus, $ \forall y \in Y, \  g(y) = g'(y)$, and therefore $g = g'$

\eproof

b) $f:\mathbb{R} \rightarrow \mathbb{R}, \ f(x) = x^2$ is not invertible because it is not a bijection. This is because $1 \neq -1$ but $f(1) = f(-1)$ (This is shown by the problem below.)

\bigskip

c) 
\bproof
$\Rightarrow$ \\
We will first prove that $f$ is a surjection. \\
By assumption $\forall y \in Y, \ f(g(y)) = y$ and because $g(y)$ is just an arbitrary element of $X$, this shows that $\forall y \in Y, \ \exists x \in X \ \text{st} \ f(x) = y$ \newline
\\
We will now show that $f$ is an injection. \\
Assume that $f$ is not an injection. \\
Thus, $\exists x, x' \in X$ st $x \neq x'$ and $f(x) = f(x')$ \\
Then $g(f(x)) = g(f(x')) = x = x'$ and thus we have reached a contradiction. \newline
\\
Thus we have proven that $f$ is bijective. \newline
\\
$\Leftarrow$ \\
Because $f:X \rightarrow Y$ is a bijective function, then \\
$\forall y \in Y, \exists ! x \in X$ st $f(x) = y$ \\
Thus we can construct a function $g: Y \rightarrow X$ that maps all $y \in Y$ to the unique $x \in X$ st $f(x) = y$ \\
Thus, by construction $g(f(x)) = x$ and $f(g(y)) = y$

\eproof

\bigskip

d) It does not follow that $f \circ g = I_Y$ \\

\bproof
Consider the function $f:\{ a \} \rightarrow \{ 1,2 \}$ such that $f(a) = 1$ 
and the function $g:\{ 1,2 \} \rightarrow \{ a \}$ such that $g(1) = a$ and $g(2) = a$ \\
Thus $(g \circ f)(a) = a$ and $(g \circ f) = I_X$
But $(f \circ g)(2) = 1$ so $(f \circ g) \neq I_Y$ \\
Thus, we have constructed a counterexample.
\eproof

e) It does follow that $f \circ g = I_Y$ now.

\bproof
If $g \circ f = I_X$ then $f$ is injective because if we assume $f$ is not injective, then we result in the contradiction where $g(f(x)) = g(f(x')) = x = x'$ when $x \neq x'$ \\
Thus $f$ is injective and surjective, making it a bijection, and by part c, this implies that it is invertible and there exists a $g$ st $f \circ g = I_Y$
\eproof

\newpage

4. a) I have italicized the elements in the domain of $f_i: \{ 1, 2, 3, 4 \} \rightarrow \{ 1, 2, 3, 4 \}$ that are fixed points.

\begin{align*}
f_1 = &
\begin{pmatrix}
\mathit{1} & \rightarrow & 1\\
\mathit{2} & \rightarrow & 2\\
\mathit{3} & \rightarrow & 3\\
\mathit{4} & \rightarrow & 4\\
\end{pmatrix} &
f_2 = & 
\begin{pmatrix}
\mathit{1} & \rightarrow & 1\\
\mathit{2} & \rightarrow & 2\\
3 & \rightarrow & 4\\
4 & \rightarrow & 3\\
\end{pmatrix} &
f_3 = & 
\begin{pmatrix}
\mathit{1} & \rightarrow & 1\\
2 & \rightarrow & 3\\
3 & \rightarrow & 4\\
4 & \rightarrow & 2\\
\end{pmatrix} \\
f_4 = &
\begin{pmatrix}
\mathit{1} & \rightarrow & 1\\
2 & \rightarrow & 3\\
3 & \rightarrow & 2\\
4 & \rightarrow & 4\\
\end{pmatrix} &
f_5 = & 
\begin{pmatrix}
\mathit{1} & \rightarrow & 1\\
2 & \rightarrow & 4\\
3 & \rightarrow & 2\\
4 & \rightarrow & 3\\
\end{pmatrix} &
f_6 = & 
\begin{pmatrix}
\mathit{1} & \rightarrow & 1\\
2 & \rightarrow & 4\\
\mathit{3} & \rightarrow & 3\\
4 & \rightarrow & 2\\
\end{pmatrix} \\
f_7 = &
\begin{pmatrix}
1 & \rightarrow & 2\\
2 & \rightarrow & 1\\
\mathit{3} & \rightarrow & 3\\
\mathit{4} & \rightarrow & 4\\
\end{pmatrix} &
f_8 = & 
\begin{pmatrix}
1 & \rightarrow & 2\\
2 & \rightarrow & 1\\
3 & \rightarrow & 4\\
4 & \rightarrow & 3\\
\end{pmatrix} &
f_9 = & 
\begin{pmatrix}
1 & \rightarrow & 2\\
2 & \rightarrow & 3\\
3 & \rightarrow & 1\\
4 & \rightarrow & 4\\
\end{pmatrix} \\
f_{10} = &
\begin{pmatrix}
1 & \rightarrow & 2\\
2 & \rightarrow & 3\\
3 & \rightarrow & 4\\
4 & \rightarrow & 1\\
\end{pmatrix} &
f_{11} = & 
\begin{pmatrix}
1 & \rightarrow & 2\\
2 & \rightarrow & 4\\
3 & \rightarrow & 1\\
4 & \rightarrow & 3\\
\end{pmatrix} &
f_{12} = & 
\begin{pmatrix}
1 & \rightarrow & 2\\
2 & \rightarrow & 4\\
\mathit{3} & \rightarrow & 3\\
4 & \rightarrow & 1\\
\end{pmatrix} \\
f_{13} = &
\begin{pmatrix}
1 & \rightarrow & 3\\
2 & \rightarrow & 1\\
3 & \rightarrow & 2\\
\mathit{4} & \rightarrow & 4\\
\end{pmatrix} &
f_{14} = & 
\begin{pmatrix}
1 & \rightarrow & 3\\
2 & \rightarrow & 1\\
3 & \rightarrow & 4\\
4 & \rightarrow & 2\\
\end{pmatrix} &
f_{15} = & 
\begin{pmatrix}
1 & \rightarrow & 3\\
\mathit{2} & \rightarrow & 2\\
3 & \rightarrow & 1\\
\mathit{4} & \rightarrow & 4\\
\end{pmatrix} \\
f_{16} = &
\begin{pmatrix}
1 & \rightarrow & 3\\
\mathit{2} & \rightarrow & 2\\
3 & \rightarrow & 4\\
4 & \rightarrow & 1\\
\end{pmatrix} &
f_{17} = & 
\begin{pmatrix}
1 & \rightarrow & 3\\
2 & \rightarrow & 4\\
3 & \rightarrow & 1\\
4 & \rightarrow & 2\\
\end{pmatrix} &
f_{18} = & 
\begin{pmatrix}
1 & \rightarrow & 3\\
2 & \rightarrow & 4\\
3 & \rightarrow & 2\\
4 & \rightarrow & 1\\
\end{pmatrix} \\
f_{19} = &
\begin{pmatrix}
1 & \rightarrow & 4\\
2 & \rightarrow & 1\\
3 & \rightarrow & 2\\
4 & \rightarrow & 3\\
\end{pmatrix} &
f_{20} = & 
\begin{pmatrix}
1 & \rightarrow & 4\\
2 & \rightarrow & 1\\
\mathit{3} & \rightarrow & 3\\
4 & \rightarrow & 2\\
\end{pmatrix} &
f_{21} = & 
\begin{pmatrix}
1 & \rightarrow & 4\\
\mathit{2} & \rightarrow & 2\\
3 & \rightarrow & 1\\
4 & \rightarrow & 3\\
\end{pmatrix} \\
f_{22} = &
\begin{pmatrix}
1 & \rightarrow & 4\\
\mathit{2} & \rightarrow & 2\\
\mathit{3} & \rightarrow & 3\\
4 & \rightarrow & 1\\
\end{pmatrix} &
f_{23} = & 
\begin{pmatrix}
1 & \rightarrow & 4\\
2 & \rightarrow & 3\\
3 & \rightarrow & 1\\
4 & \rightarrow & 2\\
\end{pmatrix} &
f_{24} = & 
\begin{pmatrix}
1 & \rightarrow & 4\\
2 & \rightarrow & 3\\
3 & \rightarrow & 2\\
4 & \rightarrow & 1\\
\end{pmatrix} \\
\end{align*}

\newpage

b) Let $M = \{ f_1, f_2, \cdots, f_{24} \}$ and let $i: M \rightarrow M, \ i(f) = $ the inverse of $f$. Again, I will italicize the elements of the domain that are fixed points.

\begin{align*}
i = &
\begin{pmatrix}
\mathit{f_1} & \rightarrow & f_1 \\
\mathit{f_2} & \rightarrow & f_2 \\
\mathit{f_3} & \rightarrow & f_3 \\
f_4 & \rightarrow & f_5 \\
f_5 & \rightarrow & f_4 \\
\mathit{f_6} & \rightarrow & f_6 \\
\mathit{f_7} & \rightarrow & f_7 \\
\mathit{f_8} & \rightarrow & f_8 \\
f_9 & \rightarrow & f_{13} \\
f_{10} & \rightarrow & f_{19} \\
f_{11} & \rightarrow & f_{14} \\
f_{12} & \rightarrow & f_{20} \\
f_{13} & \rightarrow & f_9 \\
f_{14} & \rightarrow & f_{11} \\
\mathit{f_{15}} & \rightarrow & f_{15} \\
f_{16} & \rightarrow & f_{21} \\
\mathit{f_{17}} & \rightarrow & f_{17} \\
f_{18} & \rightarrow & f_{23} \\
f_{19} & \rightarrow & f_{10} \\
f_{20} & \rightarrow & f_{12} \\
f_{21} & \rightarrow & f_{16} \\
\mathit{f_{22}} & \rightarrow & f_{22} \\
f_{23} & \rightarrow & f_{18} \\
\mathit{f_{24}} & \rightarrow & f_{24}
\end{pmatrix}
\end{align*}

\newpage

5. There are 203 distinct partitions of the set $\{ 1, 2, 3, 4, 5, 6 \}$.

\bproof

\begin{align*}
\{ \{ 1, 2, 3, 4, 5, 6 \} \} \\
\{\{1\}, \{2, 3, 4, 5, 6\}\}\\
\{\{1\}, \{2\}, \{3, 4, 5, 6\}\}\\
\{\{1\}, \{2\}, \{3\}, \{4, 5, 6\}\}\\
\{\{1\}, \{2\}, \{3\}, \{4\}, \{5, 6\}\}\\
\{\{1\}, \{2\}, \{3\}, \{4\}, \{5\}, \{6\}\}\\
\{\{1\}, \{2\}, \{3\}, \{4, 5\}, \{6\}\}\\
\{\{1\}, \{2\}, \{3\}, \{4, 6\}, \{5\}\}\\
\{\{1\}, \{2\}, \{3, 4\}, \{5, 6\}\}\\
\{\{1\}, \{2\}, \{3, 4\}, \{5\}, \{6\}\}\\
\{\{1\}, \{2\}, \{3, 5\}, \{4, 6\}\}\\
\{\{1\}, \{2\}, \{3, 5\}, \{4\}, \{6\}\}\\
\{\{1\}, \{2\}, \{3, 6\}, \{4, 5\}\}\\
\{\{1\}, \{2\}, \{3, 6\}, \{4\}, \{5\}\}\\
\{\{1\}, \{2\}, \{3, 4, 5\}, \{6\}\}\\
\{\{1\}, \{2\}, \{3, 4, 6\}, \{5\}\}\\
\{\{1\}, \{2\}, \{3, 5, 6\}, \{4\}\}\\
\{\{1\}, \{2, 3\}, \{4, 5, 6\}\}\\
\{\{1\}, \{2, 3\}, \{4\}, \{5, 6\}\}\\
\{\{1\}, \{2, 3\}, \{4\}, \{5\}, \{6\}\}\\
\{\{1\}, \{2, 3\}, \{4, 5\}, \{6\}\}\\
\{\{1\}, \{2, 3\}, \{4, 6\}, \{5\}\}\\
\{\{1\}, \{2, 4\}, \{3, 5, 6\}\}\\
\{\{1\}, \{2, 4\}, \{3\}, \{5, 6\}\}\\
\{\{1\}, \{2, 4\}, \{3\}, \{5\}, \{6\}\}\\
\{\{1\}, \{2, 4\}, \{3, 5\}, \{6\}\}\\
\{\{1\}, \{2, 4\}, \{3, 6\}, \{5\}\}\\
\end{align*}
\begin{align*}
\{\{1\}, \{2, 5\}, \{3, 4, 6\}\}\\
\{\{1\}, \{2, 5\}, \{3\}, \{4, 6\}\}\\
\{\{1\}, \{2, 5\}, \{3\}, \{4\}, \{6\}\}\\
\{\{1\}, \{2, 5\}, \{3, 4\}, \{6\}\}\\
\{\{1\}, \{2, 5\}, \{3, 6\}, \{4\}\}\\
\{\{1\}, \{2, 6\}, \{3, 4, 5\}\}\\
\{\{1\}, \{2, 6\}, \{3\}, \{4, 5\}\}\\
\{\{1\}, \{2, 6\}, \{3\}, \{4\}, \{5\}\}\\
\{\{1\}, \{2, 6\}, \{3, 4\}, \{5\}\}\\
\{\{1\}, \{2, 6\}, \{3, 5\}, \{4\}\}\\
\{\{1\}, \{2, 3, 4\}, \{5, 6\}\}\\
\{\{1\}, \{2, 3, 4\}, \{5\}, \{6\}\}\\
\{\{1\}, \{2, 3, 5\}, \{4, 6\}\}\\
\{\{1\}, \{2, 3, 5\}, \{4\}, \{6\}\}\\
\{\{1\}, \{2, 3, 6\}, \{4, 5\}\}\\
\{\{1\}, \{2, 3, 6\}, \{4\}, \{5\}\}\\
\{\{1\}, \{2, 4, 5\}, \{3, 6\}\}\\
\{\{1\}, \{2, 4, 5\}, \{3\}, \{6\}\}\\
\{\{1\}, \{2, 4, 6\}, \{3, 5\}\}\\
\{\{1\}, \{2, 4, 6\}, \{3\}, \{5\}\}\\
\{\{1\}, \{2, 5, 6\}, \{3, 4\}\}\\
\{\{1\}, \{2, 5, 6\}, \{3\}, \{4\}\}\\
\{\{1\}, \{2, 3, 4, 5\}, \{6\}\}\\
\{\{1\}, \{2, 3, 4, 6\}, \{5\}\}\\
\{\{1\}, \{2, 3, 5, 6\}, \{4\}\}\\
\{\{1\}, \{2, 4, 5, 6\}, \{3\}\}\\
\end{align*}
\begin{align*}
\{\{1, 2\}, \{3, 4, 5, 6\}\}\\
\{\{1, 2\}, \{3\}, \{4, 5, 6\}\}\\
\{\{1, 2\}, \{3\}, \{4\}, \{5, 6\}\}\\
\{\{1, 2\}, \{3\}, \{4\}, \{5\}, \{6\}\}\\
\{\{1, 2\}, \{3\}, \{4, 5\}, \{6\}\}\\
\{\{1, 2\}, \{3\}, \{4, 6\}, \{5\}\}\\
\{\{1, 2\}, \{3, 4\}, \{5, 6\}\}\\
\{\{1, 2\}, \{3, 4\}, \{5\}, \{6\}\}\\
\{\{1, 2\}, \{3, 5\}, \{4, 6\}\}\\
\{\{1, 2\}, \{3, 5\}, \{4\}, \{6\}\}\\
\{\{1, 2\}, \{3, 6\}, \{4, 5\}\}\\
\{\{1, 2\}, \{3, 6\}, \{4\}, \{5\}\}\\
\{\{1, 2\}, \{3, 4, 5\}, \{6\}\}\\
\{\{1, 2\}, \{3, 4, 6\}, \{5\}\}\\
\{\{1, 2\}, \{3, 5, 6\}, \{4\}\}\\
\{\{1, 3\}, \{2, 4, 5, 6\}\}\\
\{\{1, 3\}, \{2\}, \{4, 5, 6\}\}\\
\{\{1, 3\}, \{2\}, \{4\}, \{5, 6\}\}\\
\{\{1, 3\}, \{2\}, \{4\}, \{5\}, \{6\}\}\\
\{\{1, 3\}, \{2\}, \{4, 5\}, \{6\}\}\\
\{\{1, 3\}, \{2\}, \{4, 6\}, \{5\}\}\\
\{\{1, 3\}, \{2, 4\}, \{5, 6\}\}\\
\{\{1, 3\}, \{2, 4\}, \{5\}, \{6\}\}\\
\{\{1, 3\}, \{2, 5\}, \{4, 6\}\}\\
\{\{1, 3\}, \{2, 5\}, \{4\}, \{6\}\}\\
\{\{1, 3\}, \{2, 6\}, \{4, 5\}\}\\
\{\{1, 3\}, \{2, 6\}, \{4\}, \{5\}\}\\
\{\{1, 3\}, \{2, 4, 5\}, \{6\}\}\\
\end{align*}
\begin{align*}
\{\{1, 3\}, \{2, 4, 6\}, \{5\}\}\\
\{\{1, 3\}, \{2, 5, 6\}, \{4\}\}\\
\{\{1, 4\}, \{2, 3, 5, 6\}\}\\
\{\{1, 4\}, \{2\}, \{3, 5, 6\}\}\\
\{\{1, 4\}, \{2\}, \{3\}, \{5, 6\}\}\\
\{\{1, 4\}, \{2\}, \{3\}, \{5\}, \{6\}\}\\
\{\{1, 4\}, \{2\}, \{3, 5\}, \{6\}\}\\
\{\{1, 4\}, \{2\}, \{3, 6\}, \{5\}\}\\
\{\{1, 4\}, \{2, 3\}, \{5, 6\}\}\\
\{\{1, 4\}, \{2, 3\}, \{5\}, \{6\}\}\\
\{\{1, 4\}, \{2, 5\}, \{3, 6\}\}\\
\{\{1, 4\}, \{2, 5\}, \{3\}, \{6\}\}\\
\{\{1, 4\}, \{2, 6\}, \{3, 5\}\}\\
\{\{1, 4\}, \{2, 6\}, \{3\}, \{5\}\}\\
\{\{1, 4\}, \{2, 3, 5\}, \{6\}\}\\
\{\{1, 4\}, \{2, 3, 6\}, \{5\}\}\\
\{\{1, 4\}, \{2, 5, 6\}, \{3\}\}\\
\{\{1, 5\}, \{2, 3, 4, 6\}\}\\
\{\{1, 5\}, \{2\}, \{3, 4, 6\}\}\\
\{\{1, 5\}, \{2\}, \{3\}, \{4, 6\}\}\\
\{\{1, 5\}, \{2\}, \{3\}, \{4\}, \{6\}\}\\
\{\{1, 5\}, \{2\}, \{3, 4\}, \{6\}\}\\
\{\{1, 5\}, \{2\}, \{3, 6\}, \{4\}\}\\
\{\{1, 5\}, \{2, 3\}, \{4, 6\}\}\\
\{\{1, 5\}, \{2, 3\}, \{4\}, \{6\}\}\\
\{\{1, 5\}, \{2, 4\}, \{3, 6\}\}\\
\{\{1, 5\}, \{2, 4\}, \{3\}, \{6\}\}\\
\{\{1, 5\}, \{2, 6\}, \{3, 4\}\}\\
\end{align*}
\begin{align*}
\{\{1, 5\}, \{2, 6\}, \{3\}, \{4\}\}\\
\{\{1, 5\}, \{2, 3, 4\}, \{6\}\}\\
\{\{1, 5\}, \{2, 3, 6\}, \{4\}\}\\
\{\{1, 5\}, \{2, 4, 6\}, \{3\}\}\\
\{\{1, 6\}, \{2, 3, 4, 5\}\}\\
\{\{1, 6\}, \{2\}, \{3, 4, 5\}\}\\
\{\{1, 6\}, \{2\}, \{3\}, \{4, 5\}\}\\
\{\{1, 6\}, \{2\}, \{3\}, \{4\}, \{5\}\}\\
\{\{1, 6\}, \{2\}, \{3, 4\}, \{5\}\}\\
\{\{1, 6\}, \{2\}, \{3, 5\}, \{4\}\}\\
\{\{1, 6\}, \{2, 3\}, \{4, 5\}\}\\
\{\{1, 6\}, \{2, 3\}, \{4\}, \{5\}\}\\
\{\{1, 6\}, \{2, 4\}, \{3, 5\}\}\\
\{\{1, 6\}, \{2, 4\}, \{3\}, \{5\}\}\\
\{\{1, 6\}, \{2, 5\}, \{3, 4\}\}\\
\{\{1, 6\}, \{2, 5\}, \{3\}, \{4\}\}\\
\{\{1, 6\}, \{2, 3, 4\}, \{5\}\}\\
\{\{1, 6\}, \{2, 3, 5\}, \{4\}\}\\
\{\{1, 6\}, \{2, 4, 5\}, \{3\}\}\\
\{\{1, 2, 3\}, \{4, 5, 6\}\}\\
\{\{1, 2, 3\}, \{4\}, \{5, 6\}\}\\
\{\{1, 2, 3\}, \{4\}, \{5\}, \{6\}\}\\
\{\{1, 2, 3\}, \{4, 5\}, \{6\}\}\\
\{\{1, 2, 3\}, \{4, 6\}, \{5\}\}\\
\{\{1, 2, 4\}, \{3, 5, 6\}\}\\
\{\{1, 2, 4\}, \{3\}, \{5, 6\}\}\\
\{\{1, 2, 4\}, \{3\}, \{5\}, \{6\}\}\\
\{\{1, 2, 4\}, \{3, 5\}, \{6\}\}\\
\{\{1, 2, 4\}, \{3, 6\}, \{5\}\}\\
\end{align*}
\begin{align*}
\{\{1, 2, 5\}, \{3, 4, 6\}\}\\
\{\{1, 2, 5\}, \{3\}, \{4, 6\}\}\\
\{\{1, 2, 5\}, \{3\}, \{4\}, \{6\}\}\\
\{\{1, 2, 5\}, \{3, 4\}, \{6\}\}\\
\{\{1, 2, 5\}, \{3, 6\}, \{4\}\}\\
\{\{1, 2, 6\}, \{3, 4, 5\}\}\\
\{\{1, 2, 6\}, \{3\}, \{4, 5\}\}\\
\{\{1, 2, 6\}, \{3\}, \{4\}, \{5\}\}\\
\{\{1, 2, 6\}, \{3, 4\}, \{5\}\}\\
\{\{1, 2, 6\}, \{3, 5\}, \{4\}\}\\
\{\{1, 3, 4\}, \{2, 5, 6\}\}\\
\{\{1, 3, 4\}, \{2\}, \{5, 6\}\}\\
\{\{1, 3, 4\}, \{2\}, \{5\}, \{6\}\}\\
\{\{1, 3, 4\}, \{2, 5\}, \{6\}\}\\
\{\{1, 3, 4\}, \{2, 6\}, \{5\}\}\\
\{\{1, 3, 5\}, \{2, 4, 6\}\}\\
\{\{1, 3, 5\}, \{2\}, \{4, 6\}\}\\
\{\{1, 3, 5\}, \{2\}, \{4\}, \{6\}\}\\
\{\{1, 3, 5\}, \{2, 4\}, \{6\}\}\\
\{\{1, 3, 5\}, \{2, 6\}, \{4\}\}\\
\{\{1, 3, 6\}, \{2, 4, 5\}\}\\
\{\{1, 3, 6\}, \{2\}, \{4, 5\}\}\\
\{\{1, 3, 6\}, \{2\}, \{4\}, \{5\}\}\\
\{\{1, 3, 6\}, \{2, 4\}, \{5\}\}\\
\{\{1, 3, 6\}, \{2, 5\}, \{4\}\}\\
\{\{1, 4, 5\}, \{2, 3, 6\}\}\\
\{\{1, 4, 5\}, \{2\}, \{3, 6\}\}\\
\{\{1, 4, 5\}, \{2\}, \{3\}, \{6\}\}\\
\end{align*}
\begin{align*}
\{\{1, 4, 5\}, \{2, 3\}, \{6\}\}\\
\{\{1, 4, 5\}, \{2, 6\}, \{3\}\}\\
\{\{1, 4, 6\}, \{2, 3, 5\}\}\\
\{\{1, 4, 6\}, \{2\}, \{3, 5\}\}\\
\{\{1, 4, 6\}, \{2\}, \{3\}, \{5\}\}\\
\{\{1, 4, 6\}, \{2, 3\}, \{5\}\}\\
\{\{1, 4, 6\}, \{2, 5\}, \{3\}\}\\
\{\{1, 5, 6\}, \{2, 3, 4\}\}\\
\{\{1, 5, 6\}, \{2\}, \{3, 4\}\}\\
\{\{1, 5, 6\}, \{2\}, \{3\}, \{4\}\}\\
\{\{1, 5, 6\}, \{2, 3\}, \{4\}\}\\
\{\{1, 5, 6\}, \{2, 4\}, \{3\}\}\\
\{\{1, 2, 3, 4\}, \{5, 6\}\}\\
\{\{1, 2, 3, 4\}, \{5\}, \{6\}\}\\
\{\{1, 2, 3, 5\}, \{4, 6\}\}\\
\{\{1, 2, 3, 5\}, \{4\}, \{6\}\}\\
\{\{1, 2, 3, 6\}, \{4, 5\}\}\\
\{\{1, 2, 3, 6\}, \{4\}, \{5\}\}\\
\{\{1, 2, 4, 5\}, \{3, 6\}\}\\
\{\{1, 2, 4, 5\}, \{3\}, \{6\}\}\\
\{\{1, 2, 4, 6\}, \{3, 5\}\}\\
\{\{1, 2, 4, 6\}, \{3\}, \{5\}\}\\
\{\{1, 2, 5, 6\}, \{3, 4\}\}\\
\{\{1, 2, 5, 6\}, \{3\}, \{4\}\}\\
\{\{1, 3, 4, 5\}, \{2, 6\}\}\\
\end{align*}
\begin{align*}
\{\{1, 3, 4, 5\}, \{2\}, \{6\}\}\\
\{\{1, 3, 4, 6\}, \{2, 5\}\}\\
\{\{1, 3, 4, 6\}, \{2\}, \{5\}\}\\
\{\{1, 3, 5, 6\}, \{2, 4\}\}\\
\{\{1, 3, 5, 6\}, \{2\}, \{4\}\}\\
\{\{1, 4, 5, 6\}, \{2, 3\}\}\\
\{\{1, 4, 5, 6\}, \{2\}, \{3\}\}\\
\{\{1, 2, 3, 4, 5\}, \{6\}\}\\
\{\{1, 2, 3, 4, 6\}, \{5\}\}\\
\{\{1, 2, 3, 5, 6\}, \{4\}\}\\
\{\{1, 2, 4, 5, 6\}, \{3\}\}\\
\{\{1, 3, 4, 5, 6\}, \{2\}\}\\
\end{align*}
Thus, these are all the 203 distinct partitions of the set $\{ 1, 2, 3, 4, 5, 6 \}$.
\eproof

\newpage

6. $| \mathcal{P}(X) | = 2^n$

\bproof

Consider $X$ is a set with $n \in \mathbb{N}$ elements. Then for every element $x \in \mathcal{P}(X)$ and for every element $y \in X$, there are two choices: either $y \in x$ or $y \notin x$. For $n$ objects in $X$, this gives us $\underbrace{2 \cdot 2 \cdots 2}_\text{$n$ times}$ distinct objects in $\mathcal{P}(X)$. Hence, the cardinality of $\mathcal{P}(X)$ is $2^n$

\eproof


\end{flushleft}
\end{document}