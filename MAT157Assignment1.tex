\documentclass[11pt]{article}
\usepackage[utf8]{inputenc}
\usepackage{amssymb, amsmath, amsthm, changepage}

\title{MAT157 Assignment 1}
\author{Nicolas Coballe}

\newcommand{\bproof}{\begin{proof}
$ $ \\
\begin{adjustwidth}{3em}{0pt}
}

\newcommand{\eproof}{\end{adjustwidth}
\end{proof}}

\begin{document}

\maketitle
\begin{flushleft}

5. a) We will prove this using contradiction.

\bproof

Assume all functions $f: A \rightarrow B$ are not bijective. Then, consider $A := \mathbb{R}$ and $f:A \rightarrow A, \ f(x) = \frac{1}{x}$ if $x \neq 0$ and $f(x) = 0$ otherwise. $f$ is injective because it only maps zero to itself and strictly decreases on the interval $(- \infty ,0)$ and $(0, \infty )$ We will show this by taking $0<x<y \implies 0< \frac{x}{y} < 1 \implies 0 < \frac{1}{x} < \frac{1}{y} $. A similar argument works for $x<y<0$. Because $f$ is strictly decreasing on intervals  $(- \infty ,0)$ and $(0, \infty )$ and the images of $f((- \infty ,0) )$ and $f( (0, \infty ))$ are clearly disjoint, $f$ must be an injection. Now let $B := f(A)$ and thus, $f(A) \subseteq B \subseteq A$. Now consider the map $h: A \rightarrow B, \ f(x) = \frac{1}{x}$ if $x \neq 0$ and $f(x) = 0$ otherwise. We have already proven that this function is injective and because $h(A) = B, \  h$ is surjective. Thus $h$ is bijective, a contradicition.

\eproof


\end{flushleft}
\end{document}