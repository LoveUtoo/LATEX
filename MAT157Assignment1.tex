\documentclass[11pt]{article}
\usepackage[utf8]{inputenc}
\usepackage{amssymb, amsmath, amsthm, changepage}

\title{MAT157 Assignment 1}
\author{Nicolas Coballe}

\newcommand{\bproof}{\begin{proof}
$ $ \\
\begin{adjustwidth}{3em}{0pt}
}

\newcommand{\eproof}{\end{adjustwidth}
\end{proof}}

\begin{document}

\maketitle
\begin{flushleft}

1. a) $sin(\frac{11}{24} \pi) = \sqrt{\frac{\sqrt{\frac{\frac{\sqrt{3}}{2} + 1}{2}} + 1}{2}}$

\bproof
Consider the double angle formula for cosine. $cos(2 \theta ) = 2cos^2( \theta)-1 = 1 - 2sin^2( \theta)$  . If $ \theta = \frac{\pi}{12}$ and $ \alpha = \frac{\pi}{24}$, then we can write $cos(\frac{\pi}{6}) = 2cos^2(\frac{\pi}{12})-1$. By rearrangement:
\begin{align*}
cos( \frac{\pi}{12}) = & \sqrt{\frac{cos(\frac{\pi}{6}) + 1}{2}} \\
= &\sqrt{\frac{\frac{\sqrt{3}}{2} + 1}{2}} \\
cos( \frac{\pi}{24}) = & \sqrt{\frac{cos(\frac{\pi}{12}) + 1}{12}} \\
= &\sqrt{\frac{\sqrt{\frac{\frac{\sqrt{3}}{2} + 1}{2}} + 1}{2}} \\
\end{align*}
Because $\frac{\pi}{12}$ is in the first quadrant, we can take the positive root for $cos( \frac{\pi}{12})$.\\
\bigskip
Consider $sin( \frac{11}{24} \pi )$. $\frac{11}{24} \pi = \frac{\pi}{2} - \frac{\pi}{24}$; thus, $sin( \frac{11}{24} \pi ) = sin(\frac{\pi}{2} - \frac{\pi}{24})$. With the compound angle formula, we can write: \\
$sin( \frac{\pi}{2} - \frac{\pi}{24} ) = sin( \frac{\pi}{2} )cos (- \frac{\pi}{24}) + cos( \frac{\pi}{2} )sin (- \frac{\pi}{24})$ \\
Thus we can use our known trigonometric values, our own formula, and the fact that cosine is an even function to evaluate the statement to be: \\
\begin{align*}
 & \sqrt{\frac{\sqrt{\frac{\frac{\sqrt{3}}{2} + 1}{2}} + 1}{2}} + 0(sin(- \frac{\pi}{24})) \\
=& \sqrt{\frac{\sqrt{\frac{\frac{\sqrt{3}}{2} + 1}{2}} + 1}{2}}
\end{align*}

\eproof

b) $sin(\alpha)sin(\beta)sin(\gamma) = \frac{1}{4}sin( \alpha - \beta + \gamma) - \frac{1}{4}sin( \alpha - \beta - \gamma) - \frac{1}{4}sin( \alpha + \beta + \gamma) + \frac{1}{4}sin( \alpha + \beta - \gamma)$

\bproof

By the product formula: $sin(\alpha)sin(\beta) = \frac{1}{2}(cos( \alpha - \beta) - cos( \alpha + \beta ))$. \\
Then,
\begin{align*}
sin(\alpha)sin(\beta)sin(\gamma) = & \frac{1}{2}(cos( \alpha - \beta) - cos( \alpha + \beta ))(sin(\gamma)) \\
= & \frac{1}{2}(cos( \alpha - \beta)sin(\gamma) - cos( \alpha + \beta )sin(\gamma))
\end{align*}
Using the other product formula: $cos(\alpha)sin(\beta) = \frac{1}{2}(sin( \alpha + \beta) - sin( \alpha - \beta ))$. \\
Substituting that formula leads us to 
\begin{align*}
& \frac{1}{2}(cos( \alpha - \beta)sin(\gamma) - cos( \alpha + \beta )sin(\gamma)) \\
= & \frac{1}{2}(\frac{1}{2}(sin(\alpha - \beta + \gamma) - sin( \alpha - \beta - \gamma ))- \frac{1}{2}(sin( \alpha + \beta + \gamma ) - sin( \alpha + \beta - \gamma))) \\
= & \frac{1}{4}sin( \alpha - \beta + \gamma ) - \frac{1}{4}sin( \alpha - \beta - \gamma ) - \frac{1}{4}sin( \alpha + \beta + \gamma ) + \frac{1}{4}sin( \alpha + \beta - \gamma )
\end{align*}
\eproof

c) $sin( \omega_1 t) + sin ( \omega_2 t) = 2sin(\frac{ t (\omega_1 + \omega_2)}{2})cos(\frac{ t (\omega_1 - \omega_2)}{2})$

\bproof

Using the product formula: $sin(\alpha)cos(\beta) = \frac{1}{2}(sin( \alpha + \beta) + sin( \alpha - \beta ))$. Setting $\alpha := \frac{ t (\omega_1 + \omega_2)}{2}$ and $\beta := \frac{ t (\omega_1 - \omega_2)}{2}$, we are given $sin(\frac{ t (\omega_1 + \omega_2)}{2})cos(\frac{ t (\omega_1 - \omega_2)}{2}) = \frac{1}{2}(sin(\frac{ t (\omega_1 + \omega_2)}{2} + \frac{ t (\omega_1 - \omega_2)}{2}) + sin(\frac{ t (\omega_1 + \omega_2)}{2} - \frac{ t (\omega_1 - \omega_2)}{2}) $ \\
From here we can simplify terms:
\begin{align*}
sin(\frac{ t (\omega_1 + \omega_2)}{2})cos(\frac{ t (\omega_1 - \omega_2)}{2}) = & \frac{1}{2}(sin(\omega_1 t) + sin( \omega_2 t)) \\
2sin(\frac{ t (\omega_1 + \omega_2)}{2})cos(\frac{ t (\omega_1 - \omega_2)}{2}) = & sin(\omega_1 t) + sin( \omega_2 t)
\end{align*}
Thus we have written $sin( \omega_1 t) + sin ( \omega_2 t)$ as the product of two trigonometric functions.
\eproof

\newpage

2. a)

\bproof

Consider that $\sqrt{243}$ is rational.

\begin{align*}
\sqrt{243} = & \frac{p}{q} \\
243 = & \frac{p^2}{q^2} \\
243q^2 = & p^2 \\
3^5q^2 = & p^2 \\
\end{align*}

Because $p$ is raised to the power of 2, then it has to have an even amount of factors of 3, but because $3^5q^2$ has a even amount plus 5 factors of 3 (because an even number plus an odd number is always odd), $3^5q^2$ will never have an even amount of factors of 3, contradicting the fundamental theorem of arithmetic.

\eproof

b)

\bproof

Consider that $\sqrt[11]{11}$ is rational.

\begin{align*}
\sqrt[11]{11} = & \frac{p}{q} \\
11 = & \frac{p^{11}}{q^{11}} \\
11q^{11} = & p^{11}
\end{align*}

Because $p$ is raised to the power of 11, then it has to have a multiple of 11 factors of 11, but because $11q^{11}$ has a multiple of 11 plus 1 factors of 11 (never a multiple of 11), this contradicts the fundamental theorem of arithmetic.

\eproof

c)

\bproof

First we will show that $\sqrt{35}$ is irrational. Assume that $\sqrt{35}$ is rational first.
\begin{align*}
\sqrt{35} = & \frac{p}{q} \\
35 = & \frac{p^2}{q^2} \\
35q^2 = & p^2 \\
5\cdot7q^2 = & p^2 \\
\end{align*}
Thus, $5\cdot7q^2$ will always have a odd amount of factors of 5, while $p^2$ will always have an even amount of factors of 5, contradicting the fundamental theorem of arithmetic. \\
\bigskip

Assume $\sqrt{2} + \sqrt{5} + \sqrt{7}$ is rational. Then:
\begin{align*}
\sqrt{2} + \sqrt{5} + \sqrt{7} = & \frac{p}{q} \\
\sqrt{5} + \sqrt{7} = & \frac{p}{q} - \sqrt{2} \\
(\sqrt{5} + \sqrt{7})^2 = & (\frac{p}{q} - \sqrt{2})^2 \\
12 + 2 \sqrt{35} = & \frac{p^2}{q^2} - 2 \sqrt{2} \frac{p}{q} +2 \\
10 + 2 \sqrt{35} = & \frac{p^2}{q^2} - 2 \sqrt{2} \frac{p}{q} \\
\frac{p^2}{q^2} - 10 - 2\sqrt{35} = & 2 \sqrt{2} \frac{p}{q} \\
\end{align*}
Because $\frac{p^2}{q^2} - 10$ is clearly rational, we can rewrite it as $\frac{m}{n}$.
\begin{align*}
\frac{m}{n} - 2\sqrt{35} = & 2 \sqrt{2} \frac{p}{q} \\
(\frac{m}{n} - 2\sqrt{35})^2 = & (2 \sqrt{2} \frac{p}{q})^2 \\
\frac{m^2}{n^2} +140 - 4 \frac{m}{n} \sqrt{35} = & 8 \frac{p^2}{q^2}
\end{align*}
Because $\frac{m^2}{n^2} +140$ is clearly rational, we can rewrite it as $\frac{a}{b}$.
\begin{align*}
4 \sqrt{35} = 8 \frac{p^2}{q^2} - \frac{a}{b} \\
\sqrt{35} = 8 \frac{p^2}{4q^2} - \frac{a}{4b} \\
\end{align*}
This shows that $\sqrt{35}$ is rational, clearly a contradiction.

\eproof

\newpage

3. 1) $ \bigcup_{n \in \mathbb{N}} X_n = (-2,2]$ and $ \bigcap_{n \in \mathbb{N}} X_n = [-1,1]$
\bproof

We will prove the first statement by showing that for every $n>1, \ X_n \subseteq X_1$. The set $X_n = \{x : -1 - \frac{1}{n} < x \leq 1 + \frac{1}{n} \}$. If we let $n>1$, then $1 + \frac{1}{n} < 2$ and conversely, $-1 - \frac{1}{n} > -2 $. Thus, for every element of $x \in X_n, \ -2 < x < 2$; therefore, $\forall x \in X_n, \ x \in X_1$, making $X_n \subseteq X_1, \ \forall n>1$. Furthermore, the union of all $X_n$ and $X_1$ must just be $X_1$. \\
\bigskip
We will prove the second by proving that $x \in [-1,1] \iff x \in X_n, \ \forall n \in \mathbb{N}$. Because $[-1,1]$ is an interval, we can simply prove that the boundary of $(-1,1]$ is in all $X_n$ and then prove that if $y \notin [-1, 1]$, then there exists an $n \in \mathbb{N}$ such that $y \notin X_n$. We will take the boundaries, namely 1 and -1. Because $\frac{1}{n}$ is just some positive rational number, then $-1 - \frac{1}{n} < -1 < 1 < 1 + \frac{1}{n}, \ \forall n \in \mathbb{N}$ is trivially true. Now take some objects outside of the interval, namely $a:=1+ \epsilon$ and $b:=-1 - \epsilon, \ \forall \epsilon >0$. $\forall \epsilon, \ \exists n \in \mathbb{N}$ such that $1 + \frac{1}{n} < a$ and similarly $\exists m \in \mathbb{N}$ such that $b < -1 - \frac{1}{m}$. Thus for every object, $x \in [-1,1], \ x \in X_n, \ \forall n \in \mathbb{N}$ and for every object, $y \notin [-1,1], \exists m \in \mathbb{N}$ st $y \notin X_m$.

\eproof

2) a)

\bproof
$ \subseteq $ \\
Consider some $x \in f( \bigcup_{i \in I} S_i)$. Then, $x = f(y)$ for some $y \in  \bigcup_{i \in I} S_i$. Because $ \bigcup_{i \in I} f(S_i)$ is the union of all $f(S_i)$ for some $i \in I$, $x$ is also in $ \bigcup_{i \in I} f(S_i)$.

\bigskip
$ \supseteq $ \\
Consider some $x \in \bigcup_{i \in I} f(S_i)$. Then, there exists some $i \in I$ such that $x = f(y)$ for some $y \in S_i$. Because $f( \bigcup_{i \in I} S_i)$ is the image of $f$ on $ \bigcup_{i \in I} S_i$, $x$ is also in $ f( \bigcup_{i \in I} S_i)$

\eproof

b)

\bproof

Consider the indexing set $I := \{ 1, 2 \}$ and the sets $S_i \subseteq \{ 1,2,3 \}$ such that $i \in I$. $S_1 := \{ 1, 2 \}$ and $S_2 := \{ 2,3 \}$. Take the function $f: \{ 1,2,3 \} \rightarrow \{ 0 ,1 \}, \ f(x) = 1 \text{ if $x$ is odd and } 0$ otherwise. Thus $\bigcap_{i \in I} f(S_i) = \{ 1 \}$, but $f( \bigcap_{i \in I} S_i) =  \{ 0 \}$; therefore, $\bigcap_{i \in I} f(S_i) \neq f( \bigcap_{i \in I} S_i)$, so we have constructed an example.

\eproof

\newpage

4. a)

\bproof

By assumption, $\exists n,m \in \mathbb{Z}$ such that $a - a' = nk$ and $b - b' = mk$. We can rewrite these formulas as $a = nk + a'$ and $b = mk + b'$. We will now prove that addition is well-defined:
\begin{align*}
a+b = & (nk + a') + (mk + b') \\
= & a' + b' + k(n + m) \\
(a+b)-(a'+b') = & k(n + m)
\end{align*}
Thus, $[a+b]_k=[a'+b']_k$. We will now prove that multiplication is well defined:
\begin{align*}
ab = & (nk + a')(mk + b') \\
= & nmk^2 +  ka' + mk  b' + a'b' \\
= & k  (nmk + na' + mb') + a'b' \\
(ab) - (a'b') = & k(nmk + na' + mb')
\end{align*}
Thus, $[ab]_k = [a'b']_k$

\eproof

b)

\bproof

P1 \\
Like the integers, the integers modulo $k$ is also associative under addition.
\begin{align*}
\forall x,y,z \in \mathbb{Z}, \ [x]_k + ([y]_k + [z]_k) = & [x]_k + [y+z]_k \\
= & [x+y+z]_k \\
= & [x+y]_k + [z]_k \\
= & ([x]_k + [y]_k) + [z]_k
\end{align*} \\
\bigskip
P2 \\
There exists a 0-element, namely $[0]_k$. \\
$\forall x \in \mathbb{Z}, \ [x]_k+[0]_k = [0]_k + [x]_k = [x]_k$ \\
\bigskip
P3 \\
Addition under the integers modulo $k$ also has inverses.
$\forall x \in \mathbb{Z},\exists y \in \mathbb{Z}$ such that $[x]_k + [y]_k = [0]_k$. We can choose that element $y := -x$, and thus: $[x]_k + [-x]_k = [x-x]_k = [0]_k$
\bigskip
P4 \\
Inherited from the integers, the integers modulo k is also commutative under addition. \\
$\forall x,y \in \mathbb{Z}, \ [x]_k+[y]_k = [x+y]_k = [y+x]_k = [y]_k+[x]_k$ \\
\bigskip
P5 \\
Integers modulo $k$ is similarly associative under multiplication.
\begin{align*}
\forall x,y,z \in \mathbb{Z}, \ [x]_k \cdot ([y]_k \cdot [z]_k) = & [x]_k + [yz]_k \\
= & [xyz]_k \\
= & [xy]_k \cdot [z]_k \\
= & ([x]_k \cdot [y]_k) \cdot [z]_k
\end{align*} \\
\bigskip
P6 \\
There exists a $e$-element, namely $[1]_k$. \\
$\forall x \in \mathbb{Z}, \ [x]_k \cdot [1]_k = [1x]_k = [1]_k \cdot [x]_k = [x]_k$ \\
\bigskip
P7 \\
This is the problematic axiom. \\
\bigskip
P8 \\
Modular multiplication is also commutative. \\
$\forall x,y \in \mathbb{Z}, \ [x]_k \cdot [y]_k = [xy]_k = [yx]_k = [y]_k \cdot [x]_k$ \\
\bigskip
P9 \\
Finally, the distributive property holds true.
\begin{align*}
\forall x,y,z \in \mathbb{Z}, \ [x]_k \cdot ([y]_k + [z]_k) = & [x]_k \cdot [y+z]_k \\
= & [x(y+z)]_k \\
= & [xy + xz]_k \\
= & [xy]_k + [xz]_k
\end{align*}

\eproof
\newpage
c)
\bproof

We will prove this by showing that each non-zero element of $\mathbb{Z}_{13}$ has a multiplicative inverse.
\begin{align*}
[1]_{13} \cdot [1]_{13} & =  [1]_{13} \\
[2]_{13} \cdot [7]_{13} & =  [14]_{13} =  [1]_{13} \\
[3]_{13} \cdot [9]_{13} & =  [27]_{13} =  [1]_{13} \\
[4]_{13} \cdot [10]_{13} & =  [40]_{13} =  [1]_{13} \\
[5]_{13} \cdot [8]_{13} & =  [40]_{13} = [1]_{13} \\
[6]_{13} \cdot [11]_{13} & =  [66]_{13} =  [1]_{13} \\
[7]_{13} \cdot [2]_{13} & =  [14]_{13} =  [1]_{13} \\
[8]_{13} \cdot [5]_{13} & =  [40]_{13} =  [1]_{13} \\
[9]_{13} \cdot [3]_{13} & =  [27]_{13} =  [1]_{13} \\
[10]_{13} \cdot [4]_{13} & =  [40]_{13} =  [1]_{13} \\
[11]_{13} \cdot [6]_{13} & =  [66]_{13} =  [1]_{13} \\
[12]_{13} \cdot [12]_{13} & =  [144]_{13} =  [1]_{13} 
\end{align*}

\eproof

d)
\bproof
There exists an $a \in \{ 1, 2, 3, \cdots , 12 \}$ that satisfies the equation, namely $a = 2$.
\begin{align*}
\frac{1}{ \frac{[4]_{13}}{[3]_{13}} + \frac{[2]_{13} }{[7]_{13}} } - \frac{[3]_{13}}{[10]_{13}} = & \frac{1}{([4]_{13} \cdot [3]_{13}^{-1}) + ([2]_{13} \cdot [7]_{13}^{-1})} - [3]_{13} [10]_{13}^{-1} \\
= & \frac{1}{([4]_{13} \cdot [9]_{13}) + ([2]_{13} \cdot [2]_{13})} - [3]_{13} [4]_{13} \\
= & \frac{1}{[10]_{13} + [4]_{13}} - [12]_{13} \\
= & \frac{1}{[1]_{13}} - [12]_{13} \\
= & [1]_{13} + [-12]_{13} \\
= & [2]_{13}
\end{align*}
\eproof

\newpage

5. a) We will prove these statements using contradiction.

\bproof

Assume all functions $f: A \rightarrow B$ are not bijective. Then, consider $A := \{ 1 \}$ and $f:A \rightarrow A, \ f(x) = x$. $f$ is injective because there is only one element in the domain. Now let $B := f(A)$ and thus, $f(A) \subseteq B \subseteq A$. Now consider the map $h: A \rightarrow B, \ h(x) = f(x)$. We have already proven that this function is injective and because $h(A) = B, \  h$ is surjective. Thus $h$ is bijective, a contradiction.

\eproof

b)

\bproof

Assume all functions from $h:A \rightarrow B$ are not bijective. Consider $A := \{ 1 \} $; $B:= \{ 1 \}$; $f: A \rightarrow B, \ f(1) = 1$; and $g: B \rightarrow A, \ g(1) = 1$. These functions are both clearly injective. $f$ is also clearly bijective, contradicting the assumption that all functions $h:A \rightarrow B$ are not bijective.

\eproof

\end{flushleft}
\end{document}