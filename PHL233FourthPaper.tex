\documentclass[11pt]{article}
\usepackage[utf8]{inputenc}
\usepackage{amssymb, amsmath, amsthm, changepage, graphicx, caption, subcaption, setspace}

\graphicspath{{./images/}}

\title{How We Can Achieve Proper Mathematical Knowledge}
\author{Nicolas Coballe}

\newcommand{\bproof}{\begin{proof}
$ $ \\
\begin{adjustwidth}{3em}{0pt}
}

\setlength{\parindent}{5ex}

\newcommand{\eproof}{\end{adjustwidth}
\end{proof}}

\begin{document}

\doublespacing

\maketitle

\textit{Objects and Persons} by Trenton Merricks defends the Eliminativist ontological position by claiming that there would be no issue with the non-existence of ordinary-objects, $X$, only the existence of atoms arranged $X$-wise. His argument is that just because we see a collection of atoms arranged $X$-wise, does not imply the existence of an object, $X$ because he claims that visual, empirical evidence is not reliable. He highlights this by using the example that if statues were an object that that would mean that dog-treetops would be considered an object as well, which is non-nonsensical. So we must reject that statues are objects and simply refer to them as atoms arranged statue-wise. This position is not restrictive because one can simply syntactically replace all uses of the word "atoms arranged $X$-wise`` with $X$, allowing one's outward disposition to remain the same as folk ontologists (ordinary-object believers), while still rejecting ordinary-objects.

I think that Merrick's Eliminativist proposition is enticing because it allows one sustain an identical lifestyle to the folk ontologist, while also accepting that anything above the microscopic is not-real. However, I would argue that the choice of this position is weakly motivated. We could just as easily inherit the folk ontological position and declare that any collection of atoms arranged $X$-wise, constitutes $X$. For example, we know that statues exist because atoms arranged statue-wise exist. Now the issue that Merricks points out with this position is that dog-treetops can exist according to these conditions; nevertheless, I do not believe the existence of dog-treetops hinders any personal functionality, and moreover, I believe it is useful to be able to construct arbitrary objects out of separate spatio-collection of entities. The existence of $XY$ composed of atoms arranged $X$-wise and $Y$-wise has very similar functionality to Cartesian products in mathematics or just any product space of topological spaces. It also allows one to refer to organizations as single entities without trouble. Thus alternatively, we can reject the assertion that multiple objects cannot constitute a single object, allowing for ordinary-objects to be defined empirically and for more abstract, collections of ordinary-objects to exist.

Though, if one does not see the advantages of the folk ontological position, one can easily take the Elimimativist position and reject any ordinary-object, and their life would not be too deeply distorted.
\end{document}