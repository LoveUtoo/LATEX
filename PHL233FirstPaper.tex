\documentclass[11pt]{article}
\usepackage[utf8]{inputenc}
\usepackage{amssymb, amsmath, amsthm, changepage, graphicx, caption, subcaption, setspace}

\graphicspath{{./images/}}

\title{Sociological Factors that Influence Scientific Theory}
\author{Nicolas Coballe}

\newcommand{\bproof}{\begin{proof}
$ $ \\
\begin{adjustwidth}{3em}{0pt}
}

\setlength{\parindent}{5ex}

\newcommand{\eproof}{\end{adjustwidth}
\end{proof}}

\begin{document}

\doublespacing

\maketitle

\textit{The underdetermination of theory by data and the ``strong programme'' in the sociology of  knowledge} by Samir Okasha discusses the arguments that sociologist Bloor and Barnes present to support ``strong programme" theory, while also discussing the popular objections to said theory. ``Strong programme" is the belief that scientific data can be explained using non-unique, consistent theories; thus, there must be some sociological reason behind why scientists believe in the theories that they do. This leads towards the belief that scientific theories are ``underdetermined" because the reasoning behind believing in a scientific theory is arbitrary. Despite refuting the typical counter-claims to ``strong programme" brought up by Brown and Laudan, Okasha states that the true issue with "strong programme" is that ``Barnes and Bloor do nothing to show that the premise of the argument is true" (294). Okasha claims that Barnes and Bloor do not show existence for the uniqueness of two competing scientific theories providing explanation for a single data set. Overall, Okasha believes that it is sufficiently difficult to prove the existence of unique competing scientific theories; thus, the sociologists' argument is flimsy, and the other factors that influence the choice of theory minimize the sociological effect substantially.

Okasha's argument relies on the claim that it is hard to prove existence of competing scientific theories supporting a data set. However, I believe that there are strong social factors that decide whether or not a scientific theory is held on to or tossed out. For example, while not exactly a scientific discovery, both Newton and Leibniz discovered calculus independently of each other. Currently, it is accepted that both Mathematicians discovered calculus at the same time, but in the late 17th century, Newton used his influence and stature as a famous mathematician and physicist to propagate the belief that Leibniz stole his idea for calculus. Although the two competing calculus' were quite similar, Newton definitely used his \textit{mathematical clout} to persuade other scholars to choose his calculus. Even though this example is quite archaic, I still believe there is some merit to the anecdote because the academic system is still quite political; Well-established academics are typically more secure and protected than younger academics in terms of guaranteed research and economic stability. Thus, it does not seem improbable that if two competing theories were proposed by scientists of different social hierarchical standings, the more successful scientist may have more acceptance just because the scientist has more influence within the academic eco-system.



\end{document}