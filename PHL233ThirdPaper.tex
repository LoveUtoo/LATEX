\documentclass[11pt]{article}
\usepackage[utf8]{inputenc}
\usepackage{amssymb, amsmath, amsthm, changepage, graphicx, caption, subcaption, setspace}

\graphicspath{{./images/}}

\title{How We Can Achieve Proper Mathematical Knowledge}
\author{Nicolas Coballe}

\newcommand{\bproof}{\begin{proof}
$ $ \\
\begin{adjustwidth}{3em}{0pt}
}

\setlength{\parindent}{5ex}

\newcommand{\eproof}{\end{adjustwidth}
\end{proof}}

\begin{document}

\doublespacing

\maketitle

\textit{Mathematical Truth} by Paul Benacerraf discusses the issues that mathematics faces concerning the need for parallelism between mathematical semantics and ordinary language semantics and the need for mathematical knowledge to be of the same status as any other knowledge. Benacerraf states there does not exists an account of mathematical truth which allows for both these two principles to co-exist. In his argument, Benacerraf claims that under the standard platonistic view of mathematics (the view that mathematical entities are not physically tangible; however, they are real because they exist in a mind independent realm) along with the causal account of knowledge (the view that for $X$ to know $S$, $X$ requires causal relation between $X$ and the entities that $S$ refers to). He proposes an anecdote to illustrate his viewpoint: Hermione (an arbitrary name that Benacerraf chose) knows that the black object she is holding is a truffle because there is causal connection between Hermione and the black object. However, he claims that for mathematical knowledge, there cannot be a causal connections because people are physical things, while mathematical objects are abstract entities, unable to interact with the tangible. Because there is no physical interaction, there cannot be any causal dependence, and without this causal dependence, there is insufficient evidence to conclude that mathematical knowledge is valid knowledge. The logical world in which mathematics operates over is not isomorphic with our domain, the physical world; thus, we cannot derive knowledge from it.

I think that Benacerraf's claim that mathematical knowledge needs to be as equally valid as any other types of knowledge is too strict of a requirement. When discussing Constructive Empiricism, we choose to believe that $T$ is true because $T^*$ stating that $T$ is empirically adequate is true; however, for simplicity we simply say $T$ when $T^*$ is our correct account of the truth. In a similar sense we can simply bound our mathematical account of knowledge with the caveat that we are simply abiding by the rules given by our axioms, logic, proof-theory, model theory, metamatematics, etc. We could consider this to be a type of ruleset that governs mathematics. So instead of saying that we know a mathematical theorem $S$, we mean to say we know $S^*$ with $S^*$ being that $S$ is true under our mathematical ruleset. To illustrate further I will provide an example: Harry (keeping with the\textit{ Harry Potter} themed names) knows that one cannot spin the ball on ones finger in football/soccer; however, it is physically possible for Harry to spin the ball on his finger, and there is no explicit rule saying the he cannot. Harry knows that he cannot spin the ball on his finger in football because it is a logical extension of the accepted football rule that one cannot touch the ball with one's hand. Obviously, knowing that you cannot spin a football on one's finger can be valid knowledge, so that leads us to believe that mathematical theorems are valid knowledge because they are simply logical extensions of our mathematical ruleset. Thus, we can say we know something in mathematics if we simply preface it with ``we know this under the ruleset of mathematics."

However, it is important to note that this view sacrifices the realism of mathematics, accepting a formalist-type account of math. With this formalist view it is much easier to say that we have mathematical knowledge, but we just have to give up the correspondence to reality.


\end{document}